% !TeX root=main.tex
\chapter{مراجع و مدیریت آنها در لاتک}
\label{app:refMan}

\thispagestyle{empty}

\section{استفاده از مراجع و نقل‌قول‌ها}
\label{sec:refUsage}
منابعِ پایان‌نامه، پایه و اساس تحقیق شما به حساب می‌آیند و ضرورت انجام مطالعه و روش‌های به کار رفته در بسیاری از قسمت‌های آن، به کمک منابع صورت می‌گیرد. در استفاده از مراجع علمی در پایان‌نامه، باید سعی کنید بیشتر از
\textbf{منابع چاپ‌شده و مهم}
استفاده کنید و
\emph{ارجاع به داده‌های چاپ نشده، خلاصه‌ها و پایان‌نامه‌ها، سبب به‌هم‌خوردگی و کاهش اعتبار قسمت ارجاع منابع می‌شود.}
استفاده از منابع و نقل قول‌هایی به تحقیق شما ارزش می‌دهند که
\textbf{در راستای هدف تحقیق بوده و به آن اعتبار ببخشند.}
برخی از دانش‌جویان تصوّر می‌کنند که کثرت نقل‌قول‌ها و ارجاعات زیاد، مهم‌ترین معیار علمی شدن پایان‌نامه است؛ حال آنکه استناد به تعداد کثیری از منابع بدون مطالعه دقیق آنها و استفادهٔ مستقیم در پایان‌نامه، می‌تواند نشان‌دهندهٔ عدم احساس امنیت نویسنده باشد!

دو روش برای استفاده از نتایج، جملات، داده‌ها و روش‌های دیگران وجود دارد. یکی نقل‌قول مستقیم و دقیق است و دیگری استفاده غیرمستقیم در متن مقاله، که در ادامه به قواعد این دو نوع نقل‌قول و ارجاع‌دهی اشاره می‌کنیم:
\begin{description}
	\item[نقل‌قول مستقیم:]
	نقل‌قول مستقیم باید دقیق و بدون هیچ تغییری در جملات باشد. بهتر است این‌گونه نقل‌قول‌ها تا حد امکان کوتاه باشد. جملات کوتاه داخل گیومه قرار می‌گیرند و باید به منبع دقیق آن، طبق روش ارجاع‌دهی در منابع، اشاره شود.
	\item[نقل‌قول غیرمستقیم:]
	نقل‌قول غیرمستقیم به معنی استفاده از ایده‌ها، نتایج، روش‌ها و داده‌های دیگران در درون متنِ مقاله، ولی به سبک خودتان و متناسب و هماهنگ با روند مقاله شماست. در این حالت نیز باید متناسب با شیوهٔ ارجاع‌دهی به آن استناد شود.
\end{description}

لیست منابع و مراجع مورد استفاده، بایستی با شکلی همسان و با استفاده از فرمت رفرنس‌دهی
\lr{IEEE}%
\footnote{سبک رفرنس‌دهی \lr{IEEE} از لینک \url{http://www.ieee.org/documents/ieeecitationref.pdf} قابل دسترسی است.}
(برای پایان‌نامه‌های مهندسی)
در متن و در انتهای گزارش بیایند. باید تناظر یک به یک بین فهرست منابع در انتهای گزارش و منابع مورد استفاده در متن باشد.

برای سهولت مدیریت مراجعِ \پ%
، اکیداً توصیه می‌شود از یک ابزار «مدیریت منابع» (با خروجی
\texorpdfstring{\lr{Bib\TeX}}{Bib\TeX}%
) همچون
\lr{Mendeley}،
\lr{Zotero},
\lr{EndNote}
یا
\lr{Citavi}
استفاده کنید.

\section{ مدیریت مراجع با  \texorpdfstring{\lr{Bib\TeX}}{Bib\TeX} }
در بخش \ref{Sec:Ref} اشاره شد که با دستور 
 \lr{\textbackslash bibitem}
  می‌توان یک مرجع را تعریف نمود و با فرمان
 \lr{\textbackslash cite}
  به آن ارجاع داد. این روش برای تعداد مراجع زیاد و تغییرات آنها مناسب نیست. در ادامه به صورت مختصر توضیحی در خصوص برنامه \lr{BibTeX} که همراه با توزیع‌های معروف تِک عرضه می‌شود و نحوه استفاده از آن در زی‌پرشین خواهیم داشت.

یکی از روش‌های قدرتمند و انعطاف‌پذیر برای نوشتن مراجعِ مقالات و مدیریت مراجع در لاتک، استفاده از  \lr{BibTeX} است.
 روش کار با بیب‌تک به این صورت است که مجموعهٔ همهٔ مراجعی را که در \پ استفاده کرده یا خواهیم کرد، 
در پروندهٔ جداگانه‌ای نوشته و به آن فایل در سند خودمان به صورت مناسب لینک می‌دهیم.
 کنفرانس‌ها یا مجله‌های گوناگون برای نوشتن مراجع، قالب‌ها یا قراردادهای متفاوتی دارند که به آنها استیل‌های مراجع گفته می‌شود.
 در این حالت به کمک ‌استیل‌های بیب‌تک خواهید توانست تنها با تغییر یک پارامتر در پرونده‌ی ورودی خود، مراجع را مطابق قالب موردنظر تنظیم کنید. 
 بیشتر مجلات و کنفرانس‌های معتبر یک فایل سبک
 (\lr{BibTeX Style})
با پسوند \lr{bst} در وب‌گاه خود می‌گذارند که برای همین منظور طراحی شده است.

به جز نوشتن مقالات این سبک‌ها کمک بسیار خوبی برای تهیه‌ی مستندات علمی همچون پایان‌نامه‌هاست که فرد می‌تواند هر قسمت از کارش را که نوشت مراجع مربوطه را به بانک مراجع خود اضافه نماید. با داشتن چنین بانکی از مراجع، وی خواهد توانست به راحتی یک یا چند ارجاع به مراجع و یا یک یا چند بخش را حذف یا اضافه ‌نماید؛ 
مراجع به صورت خودکار مرتب شده و
\textbf{فقط مراجع ارجاع داده شده در قسمت کتاب‌نامه خواهندآمد.}
قالب مراجع به صورت یکدست مطابق سبک داده شده بوده و نیازی نیست که کاربر درگیر قالب‌دهی به مراجع باشد. 

\subsection{سبک‌های فعلی قابل استفاده در زی‌پرشین}
در اینجا مجموعه‌ سبک‌های بسته
\lr{Persian-bib}%
\footnote{ برای اطلاع بیشتر به راهنمای بستهٔ
\lr{Persian-bib}
مراجعه فرمایید.}
که برای  زی‌پرشین آماده شده‌اند به صورت مختصر معرفی شده و روش کار با آن‌ها گفته می‌شود:

\singlespacing
\begin{description}
\item [unsrt-fa.bst] این سبک متناظر با \lr{unsrt.bst} می‌باشد. مراجع به ترتیب ارجاع در متن ظاهر می‌شوند.
\item [plain-fa.bst] این سبک متناظر با \lr{plain.bst} می‌باشد. مراجع بر اساس نام‌خانوادگی نویسندگان، به ترتیب صعودی مرتب می‌شوند.
 همچنین ابتدا مراجع فارسی و سپس مراجع انگلیسی خواهند آمد.
\item [acm-fa.bst] این سبک متناظر با \lr{acm.bst} می‌باشد. شبیه \lr{plain-fa.bst} است.  قالب مراجع کمی متفاوت است. اسامی نویسندگان انگلیسی با حروف بزرگ انگلیسی نمایش داده می‌شوند. (مراجع مرتب می‌شوند)
\item [ieeetr-fa.bst] این سبک متناظر با \lr{ieeetr.bst} می‌باشد. (مراجع مرتب نمی‌شوند)
\item [plainnat-fa.bst] این سبک متناظر با \lr{plainnat.bst} می‌باشد. نیاز به بستهٔ \lr{natbib} دارد. (مراجع مرتب می‌شوند)
\item [chicago-fa.bst] این سبک متناظر با \lr{chicago.bst} می‌باشد. نیاز به بستهٔ \lr{natbib} دارد. (مراجع مرتب می‌شوند)
\item [asa-fa.bst] این سبک متناظر با \lr{asa.bst} می‌باشد. نیاز به بستهٔ \lr{natbib} دارد. (مراجع مرتب می‌شوند)
\end{description}
\doublespacing

طبق «دستورالعمل نگارش و تدوین پایان‌نامه» دانشگاه تهران
\cite{UTThesisGuide}،
ارجاع در متن می‌تواند مطابق با هر یک از دو الگوی هاروارد یا ونکوور باشد:
\singlespacing
\begin{description}
	\item[سبک هاروارد:]
	ذکر نام نویسنده و سال نشر در متن. در این الگو مراجع بر اساس حروف الفبا تنظیم می‌گردند.
	\item[سبک ونکوور:]
	ارجاع به مراجع به کمک شماره در متن. در این الگو شماره هر مرجع به ترتیب ظاهر شدن آن در متن در داخل کروشه قرار می‌گیرد. فهرست مراجع نیز بر اساس شماره مرجع (نه حروف الفبا) تنظیم می‌گردد.
\end{description}
\doublespacing
اما در تمپلت ورد
\cite{UTThesisGuide}،
به صراحت ذکر شده که برای پایان‌نامه‌های مهندسی بهتر است از سبک 
\lr{IEEE}
استفاده شود (که گونه‌ای از سبک‌های ونکوور است).
بنابراین برای این قالب پایان‌نامه، ما از استیل
\lr{ieeetr-fa}
در فایل
\lr{main.tex}
استفاده کردیم که خروجی آنرا در بخش مراجع می‌توانید مشاهده کنید. البته بهتر است بسته به حوزه پایان‌نامه، در مورد انتخاب سبک مراجع با استاد راهنمای خود نیز مشورت نمائید.

با استفاده از استیل‌های فوق می‌توانید به انواع مختلفی از مراجع فارسی و لاتین ارجاع دهید.
به عنوان مثال‌هایی از
\textbf{مراجع انگلیسی}،
مرجع
\cite{Baker02limits}
مقالهٔ یک ژورنال، مرجع
\cite{Amintoosi09video}
مقالهٔ یک کنفرانس، مرجع
\cite{Gonzalez02book}
یک کتاب، مرجع
\cite{Khalighi07MscThesis}
پایان‌نامهٔ کارشناسی ارشد و مرجع
\cite{Borman04thesis}
یک رسالهٔ دکتری می‌باشد.\\
همچنین در میان
\textbf{مراجع فارسی}،
مرجع
\cite{Vahedi87}
مقالهٔ یک مجله، مرجع
\cite{Amintoosi87afzayesh}
مقالهٔ یک کنفرانس، مرجع
\cite{Pedram80osool}
یک کتاب ترجمه‌شده با ذکر مترجمان و ویراستاران، مرجع
\cite{Pourmousa88mscThesis}
پایان‌نامهٔ کارشناسی ارشد%
\footnote{همان‌طور که در بخش
\ref{sec:refUsage}
اشاره شد، بهتر است زیاد از پایان‌نامه‌ها در مراجع استفاده نکنید.}،
مرجع
\cite{Omidali82phdThesis}
یک رسالهٔ دکتری و مراجع
\cite{persianbib87userguide, Khalighi87xepersian}
نمونه‌های متفرقه هستند.

\subsection{ نحوه استفاده از سبک‌های فارسی}
برای استفاده از بیب‌تک باید مراجع خود را در یک فایل با پسوند \lr{bib} ذخیره نمایید. یک فایل \lr{bib} در واقع یک پایگاه داده از مراجع%
\LTRfootnote{Bibliography Database}
شماست که هر مرجع در آن به عنوان یک رکورد از این پایگاه داده
با قالبی خاص ذخیره می‌شود. به هر رکورد یک مدخل%
\LTRfootnote{Entry}
گفته می‌شود. یک نمونه مدخل برای معرفی کتاب \lr{Digital Image Processing} در ادامه آمده است:

\singlespacing
\begin{LTR}
\begin{verbatim}
@BOOK{Gonzalez02image,
  AUTHOR     = {Gonzalez,, Rafael C. and Woods,, Richard E.},
  TITLE      = {Digital Image Processing},
  PUBLISHER  = {Prentice-Hall, Inc.},
  YEAR       = {2006},
  ISBN       = {013168728X},
  EDITION    = {3rd},
  ADDRESS    = {Upper Saddle River, NJ, USA}
}
\end{verbatim}
\end{LTR}
\doublespacing

در مثال فوق، \lr{@BOOK} مشخصهٔ شروع یک مدخل مربوط به یک کتاب و \lr{Gonzalez02book} برچسبی است که به این مرجع منتسب شده است.
 این برچسب بایستی یکتا باشد. برای آنکه بتوان
\textbf{برچسب مراجع}
 را به راحتی به خاطر سپرد و حتی‌الامکان برچسب‌ها متفاوت با هم باشند، معمولاً از قوانین خاصی به این منظور استفاده می‌شود. یک قانون می‌تواند
\textbf{فامیل نویسنده اول + دورقم سال نشر + اولین کلمهٔ عنوان اثر}
باشد. به
\lr{AUTHOR}، \lr{TITLE}، $\dots$ و \lr{ADDRESS}
فیلدهای این مدخل گفته می‌شود؛ که هر یک با مقادیر مربوط به مرجع مقدار گرفته‌اند. ترتیب فیلدها مهم نیست. 

انواع متنوعی از مدخل‌ها برای اقسام مختلف مراجع همچون کتاب، مقالهٔ کنفرانس و مقالهٔ ژورنال وجود دارد که برخی فیلدهای آنها با هم متفاوت است. 
نام فیلدها بیانگر نوع اطلاعات آن می‌باشد. مثالهای ذکر شده در فایل \lr{MyReferences.bib} کمک خوبی برای شما خواهد بود. 
%این فایل یک فایل متنی بوده و با ویرایشگرهای معمول همچون \lr{Notepad++} قابل ویرایش می‌باشد. برنامه‌هایی همچون 
%\lr{TeXMaker}
% امکاناتی برای نوشتن این مدخل‌ها دارند و به صورت خودکار فیلدهای مربوطه را در فایل \lr{bib}  شما قرار می‌دهند.  
با استفاده از سبک‌های فارسی آماده شده، محتویات هر فیلد می‌تواند به فارسی نوشته شود؛ ترتیب مراجع و نحوهٔ چینش فیلدهای هر مرجع را سبک مورد استفاده  مشخص خواهد کرد.

\textbf{در فایل 
\lr{MyReferences.bib}
 که همراه با این \پ هست، مثال‌های مختلفی از مراجع آمده‌اند که برای درج مراجع خود، تنها کافیست مراجع‌تان را جایگزین موارد مندرج در آن نمایید.
}

پس از قرار دادن مراجع خود، برای ساخت فایل خروجی می‌توانید دستور زیر را (در ترمینال یا از طریق \lr{Texmaker}) اجرا کنید:

\singlespacing
\begin{LTR}
	\begin{verbatim}
		latexmk -e "$pdflatex=q/xelatex -synctex=1 -interaction=nonstopmode/"
		        -pdf man.tex
	\end{verbatim}
\end{LTR}
\doublespacing
ابزار \lr{latexmk} مراحل مختلف ساخت خروجی لاتک را به طور خودکار و بهینه انجام می‌دهد و هر بار فقط مراحلی را که لازم باشد تکرار می‌کند.
روش دستی‌تر این است که یک بار \lr{XeLaTeX} را روی سند خود اجرا نمایید، سپس \lr{bibtex} و پس از آن هم ۲ بار \lr{XeLaTeX} را. در \lr{TeXMaker} کلید \lr{F11} و در \lr{TeXWorks} هم گزینه‌ی \lr{BibTeX} از منوی \lr{Typeset}، \lr{BibTeX} را روی سند شما اجرا می‌کنند.

برای بسیاری از مقالات لاتین حتی لازم نیست که مدخل مربوط به آنرا خودتان بنویسید. با جستجوی 
\textbf{نام مقاله + کلمه
\lr{bibtex}}
در اینترنت سایت‌های بسیاری همچون \lr{ACM} و \lr{ScienceDirect} را خواهید یافت که مدخل \lr{bibtex} مربوط به مقاله شما را دارند و کافیست آنرا به انتهای فایل \lr{MyReferences} اضافه کنید.
