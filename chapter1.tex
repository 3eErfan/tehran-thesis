% !TeX root=main.tex
% دستور زیر باید در اولین فصل شما باشد. آن را حذف نکنید!
\pagenumbering{arabic}

\chapter{مقدمه}
\thispagestyle{empty}
\section{آشنایی با این راهنما}
حروف‌چینی پروژه کارشناسی، پایان‌نامه یا رساله یکی از موارد پرکاربرد استفاده از
\lr{\LaTeX}
و زی‌پرشین
\cite{Khalighi87xepersian}
است. یک پروژه، پایان‌نامه یا رساله، احتیاج به تنظیمات زیادی از نظر صفحه‌آرایی دارد که وقت زیادی از دانشجو می‌گیرد. به دلیل قابلیت‌های بسیار لاتک در حروف‌چینی، کلاسی با نام 
\lr{tehran-thesis}
برای حروف‌چینی پروژه‌ها، پایان‌نامه‌ها و رساله‌های دانشگاه تهران، بر مبنای کلاس مشابه
\lr{IUST-Thesis}
تهیه شده است. این کلاس و فایل‌های همراه آن به گونه‌ای طراحی شده است که مطابق با دستورالعمل نگارش و تدوین پایان‌نامه کارشناسی ارشد و دکتری پردیس دانشکده‌های فنی دانشگاه تهران
\cite{UTThesisGuide}
باشد.

دستورالعمل نگارش و تدوین پایان‌نامه دانشگاه تهران به دو مقوله می‌پردازد، اول قالب و چگونگی صفحه‌آرایی پایان‌نامه، مانند اندازه و نوع قلم بخشهای مختلف، چینش فصلها، قالب مراجع و مواردی از این قبیل و دوم محتوای هر فصل پایان‌نامه. 
درصورت استفاده از این کلاس، نیازی نیست که دانشجو نگران مقوله اول باشد و پس از تایپ مطالب خود می‌تواند آنها را با لاتک و ابزار آن اجرا کند تا پایان‌نامه خود را با قالب دانشگاه داشته باشد. همچنین با خواندن این راهنما از ملزومات محتوایی هر فصل پایان‌نامه نیز مطلع خواهد شد.

در ادامهٔ  مقدمهٔ این راهنما، ابتدا چگونگی استفاده از کلاس پایان‌نامه و فایل‌های همراه آن را به صورت فنی شرح می‌دهیم و سپس مطالبی را در مورد ویژگی‌های محتوایی فصل ۱ پایان‌نامه (یعنی مقدمه) خواهیم آورد.
بقیهٔ فصل‌های این راهنما، تنها خصوصیات محتوایی فصول مختلف پایان‌نامه را شرح خواهند داد. نهایتاً جهت یادآوری، در پیوست‌ها مطالبی دربارهٔ آشنایی با دستورات لاتک، مدیریت مراجع در لاتک و چگونگی رسم جداول، نمودارها و الگوریتم‌ها آورده خواهند شد.

\section{چگونگی استفاده از کلاس پایان‌نامه}
کلیه فایل‌های لازم برای حروف‌چینی با کلاس فوق، داخل پوشه‌ای به نام
\lr{tehran-thesis}
قرار داده شده است. توجه داشته باشید که برای استفاده از این کلاس باید فونت‌های
\lr{IRLotusICEE}
و
\lr{IRTitr}
را داشته باشید (که همراه با این کلاس هست و نیاز به نصب نیست).
قلم‌های
\lr{IRLotusICEE}
مستخرج از قلم‌های
\lr{IRLotus}
شورای عالی اطلاع‌رسانی هستند که توسط دکتر بابایی‌زاده اصلاحاتی روی آنها صورت پذیرفته است: تبدیل صفر توپر به صفر توخالی (جهت تمایز بیشتر با نقطه) و اضافه شدن
\textit{\textbf{حالت توپر و ایرانیک توأم}}،
که این موارد در قلم‌های شورای عالی اطلاع‌رسانی وجود ندارد.

\subsection{این همه فایل؟!}
\label{muchFiles}
از آنجایی که یک پایان‌نامه یا رساله، یک نوشته بلند محسوب می‌شود، لذا اگر همه تنظیمات و مطالب پایان‌نامه را داخل یک فایل قرار بدهیم، باعث شلوغی و سردرگمی می‌شود. به همین خاطر، قسمت‌های مختلف پایان‌نامه یا رساله  داخل فایل‌های جداگانه قرار گرفته است. مثلاً تنظیمات پایه‌ای کلاس داخل فایل
\lr{tehran-thesis.cls}، 
قسمت مشخصات فارسی پایان‌نامه داخل 
\lr{faTitle.tex}،
مطالب فصل اول داخل 
\lr{chapter1.tex}
و تنظیمات قابل تغییر توسط کاربر داخل 
\lr{commands.tex}،
قرار داده شده است.
\textbf{
	فایل اصلی این مجموعه، فایل
	\lr{main.tex}
	می‌باشد.
}
% یعنی بعد از تغییر فایل‌های دیگر، برای دیدن نتیجه تغییرات، باید این فایل را اجرا کرد. بقیه فایل‌ها به این فایل، کمک می‌کنند تا بتوانیم خروجی کار را ببینیم.
اگر به فایل 
\lr{main.tex}
دقت کنید، متوجه می‌شوید که قسمت‌های مختلف پایان‌نامه، توسط دستورهایی مانند 
\lr{input}
و
\lr{include}
به فایل اصلی، یعنی 
\lr{main.tex}
معرفی شده‌اند.
با توجه به ساختار محتوایی دستورالعمل، در فایل
\lr{main.tex}
فرض شده که پایان‌نامه یا رساله شما، از ۵ فصل و تعدادی پیوست تشکیل شده است. با اینحال، شما می‌توانید به راحتی فصل‌ها و پیوست‌ها را با صلاحدید اساتید راهنما، کم و زیاد کنید. این کار، بسیار ساده است. فرض کنید بخواهید یک فصل دیگر هم به پایان‌نامه اضافه کنید. برای این کار، کافی است یک فایل با نام دلخواه مثلاً 
\lr{chapter6}
و با پسوند 
\lr{.tex}
بسازید و آن را داخل پوشه 
\lr{tehran-thesis}
قرار دهید و سپس این فایل را با دستور 
\verb!\include{chapter6}!
داخل فایل
\lr{main.tex}
 فراخوانی کنید.

\subsection{از کجا شروع کنم؟}
قبل از هر چیز، باید یک توزیع تِک مناسب مانند تک‌لایو
\lr{(TeXLive)}
را روی سیستم خود نصب کنید. تک‌لایو  را می‌توانید از 
 \href{http://www.tug.org/texlive}{سایت رسمی آن}%
\LTRfootnote{\lr{\url{http://www.tug.org/texlive}}}
 دانلود کنید یا مستقیماً از مخازن توزیع لینوکس خود بگیرید (مثلاً در اوبونتو با دستور
\LRE{\verb!sudo apt install texlive-full!}).
برای نصب تک‌لایو و اجرای اسناد زی‌پرشین می‌توانید از
\href{http://parsilatex.com/site/shop/}{دی‌وی‌دی مجموعه پارسی‌لاتک}%
\LTRfootnote{\lr{\url{http://parsilatex.com/site/shop/}}}
و فایل راهنمای موجود در آن هم کمک بگیرید.

برای تایپ و پردازش اسناد لاتک باید از یک ویرایشگر مناسب استفاده کنید. ویرایشگرهای
\lr{TeXWroks},
\lr{TeXstudio},
\lr{Texmaker}
و
\lr{BiDiTeXmaker}
بدین منظور تولید شده‌اند. می‌توان ویرایش‌گر 
 \href{https://bitbucket.org/srazi/biditexmaker3}{\lr{BiDiTeXmaker}}%
 \LTRfootnote{\lr{\url{https://bitbucket.org/srazi/biditexmaker3}}}
را که بویژه برای کار با زی‌پرشین و مطالب دوجهته بهبود یافته است، بهینه‌ترین ویرایشگر لاتک برای کار با اسناد فارسی عنوان کرد.
 
حال اگر نوشتن \پ اولین تجربه شما از کار با لاتک است، توصیه می‌شود که یک‌بار به صورت اجمالی، کتاب «%
\href{http://www.tug.ctan.org/tex-archive/info/lshort/persian/lshort.pdf}{مقدمه‌ای نه چندان کوتاه بر
\lr{\LaTeXe}}%
\LTRfootnote{\lr{\url{http://www.tug.ctan.org/tex-archive/info/lshort/persian/lshort.pdf}\hfill}}»
ترجمه دکتر مهدی امیدعلی را مطالعه کنید. این کتاب، کتاب بسیار کاملی است که خیلی از نیازهای شما در ارتباط با حروف‌چینی را برطرف می‌کند.
اگر تک لایو کامل را داشته باشید، این کتاب را هم دارید. کافیست در خط فرمان دستور زیر را بزنید:
\begin{latin}
	\texttt{texdoc lshort-persian}
\end{latin}
اگر عجله دارید، برخی دستورات پایه‌ای مورد نیاز در پیوست \ref{App:latexIntro} بیان شده‌اند.
 
بعد از موارد گفته شده، فایل 
\lr{main.tex}
و
\lr{faTitle.tex}
را باز کنید و مشخصات پایان‌نامه خود مثل نام، نام خانوادگی، عنوان پایان‌نامه و ... را جایگزین مشخصات موجود در فایل
\lr{faTitle.tex}
 کنید. نیازی نیست نگران چینش این مشخصات در فایل پی‌دی‌اف خروجی باشید، زیرا کلاس 
\lr{tehran-thesis}
همه این کارها را بطور خودکار برای شما انجام می‌دهد. در ضمن، موقع تغییر دادن دستورهای داخل فایل
\lr{faTitle.tex}
 کاملاً دقت کنید؛ این دستورها، خیلی حساس هستند و ممکن است با یک تغییر کوچک، موقع اجرا، خطا بگیرید. برای دیدن خروجی کار، فایل 
\lr{faTitle.tex}
 را 
\lr{Save}
(نه 
\lr{Save As})
کنید و بعد به فایل 
\lr{main.tex}
برگشته و آن را اجرا کنید%
\footnote{
	البته فایلهای این مجموعه به گونه‌ای هستند که در
	\lr{TeXWorks} یا
	\lr{TeXstudio}
	بدون بازگشت به فایل اصلی، می‌توانید سند خود را اجرا کنید.
}.
 حال اگر می‌خواهید مشخصات انگلیسی \پ را هم عوض کنید، فایل 
\lr{enTitle.tex}
را باز کنید و مشخصات داخلش را تغییر دهید.
%\RTLfootnote{
%برای نوشتن پروژه کارشناسی، نیازی به وارد کردن مشخصات انگلیسی پروژه نیست. بنابراین، این مشخصات بطور خودکار، نادیده گرفته می‌شود.
%}
در اینجا هم برای دیدن خروجی باید این فایل را ذخیره کرده، بعد به فایل 
\lr{main.tex}
برگشته و آن را اجرا کرد.

برای راحتی بیشتر، کلاس 
\lr{tehran-thesis.cls}
طوری طراحی شده است که کافی است فقط  یک‌بار مشخصات \پ را (در فایل‌های
\lr{faTitle.tex}
و
\lr{enTitle.tex})
وارد کنید و هر جای دیگر که این مشخصات لازم باشند، به طور خودکار درج می‌شوند. با این حال، اگر مایل بودید، می‌توانید تنظیمات موجود را تغییر دهید؛ گرچه، در صورتیکه کاربر مبتدی هستید و یا با ساختار فایل‌های  
\lr{cls}
 آشنایی ندارید، بهتر است به فایل 
\lr{tehran-thesis.cls}
دست نزنید.

نکته دیگری که باید به آن توجه کنید این است که در قالب آماده شده، سه گزینه به نام‌های
\lr{bsc}،
\lr{msc}
و
\lr{phd}
برای نوشتن پروژه، پایان‌نامه و رساله، در نظر گرفته شده است. بنابراین اگر قصد تایپ پروژهٔ کارشناسی، پایان‌نامهٔ کارشناسی ارشد یا رسالهٔ دکتری را دارید، به ترتیب باید از گزینه‌های
\lr{bsc}،
\lr{msc}
و
\lr{phd}
در فایل 
\lr{main.tex}
استفاده کنید. با انتخاب هر کدام از این گزینه‌ها، تنظیمات مربوط به آنها به طور خودکار، اعمال می‌شود.


\subsection[مطالب پایان‌نامه را چطور بنویسم؟]
{مطالب \پ را چطور بنویسم؟}
\subsubsection{نوشتن فصل‌ها}
همان‌طور که در بخش \ref{muchFiles} گفته شد برای جلوگیری از شلوغی، قسمت‌های مختلف \پ از جمله فصل‌ها، در فایل‌های جداگانه‌ای قرار داده شده‌اند. 
مثلاً اگر می‌خواهید مطالب فصل ۱ را تایپ کنید، باید فایل‌های 
\lr{main.tex}
و
\lr{chapter1.tex}
را باز کرده و مطالب خود را جایگزین محتویات داخل 
\lr{chapter1.tex}
نمایید. دقت شود که در ابتدای برخی فایلها دستوراتی نوشته شده است و از شما خواسته شده که آن دستورات را حذف نکنید.

%توجه کنید که همان‌طور که قبلاً هم گفته شد، تنها فایل قابل اجرا، 
%\lr{main.tex}
%است. لذا برای دیدن حاصل (خروجی) فایل خود، باید  
%\lr{chapter1.tex}
%را ذخیره کرده و سپس فایل 
%\lr{main.tex}
%را اجرا کنید.
%یک نکته بدیهی این است که لازم نیست فصل‌های \پ را به ترتیب تایپ کنید. می‌توانید ابتدا مطالب فصل ۳ را تایپ نموده و سپس مطالب فصل ۱ را تایپ کنید. 

نکته بسیار مهمی که در اینجا باید گفته شود این است که سیستم \lr{\TeX}، محتویات یک فایل تِک را به ترتیب پردازش می‌کند.  بنابراین، اگر مثلاً  دو فصل اول خود را نوشته و خروجی آنها را دیده‌اید و مشغول تایپ مطالب فصل ۳ هستید، بهتر است
که دو دستور 
\verb!% !TeX root=main.tex
% دستور زیر باید در اولین فصل شما باشد. آن را حذف نکنید!
\pagenumbering{arabic}

\chapter{مقدمه}
\thispagestyle{empty}
\section{آشنایی با این راهنما}
حروف‌چینی پروژه کارشناسی، پایان‌نامه یا رساله یکی از موارد پرکاربرد استفاده از
\lr{\LaTeX}
و زی‌پرشین
\cite{Khalighi87xepersian}
است. یک پروژه، پایان‌نامه یا رساله، احتیاج به تنظیمات زیادی از نظر صفحه‌آرایی دارد که وقت زیادی از دانشجو می‌گیرد. به دلیل قابلیت‌های بسیار لاتک در حروف‌چینی، کلاسی با نام 
\lr{tehran-thesis}
برای حروف‌چینی پروژه‌ها، پایان‌نامه‌ها و رساله‌های دانشگاه تهران، بر مبنای کلاس مشابه
\lr{IUST-Thesis}
تهیه شده است. این کلاس و فایل‌های همراه آن به گونه‌ای طراحی شده است که مطابق با دستورالعمل نگارش و تدوین پایان‌نامه کارشناسی ارشد و دکتری پردیس دانشکده‌های فنی دانشگاه تهران
\cite{UTThesisGuide}
باشد.

دستورالعمل نگارش و تدوین پایان‌نامه دانشگاه تهران به دو مقوله می‌پردازد، اول قالب و چگونگی صفحه‌آرایی پایان‌نامه، مانند اندازه و نوع قلم بخشهای مختلف، چینش فصلها، قالب مراجع و مواردی از این قبیل و دوم محتوای هر فصل پایان‌نامه. 
درصورت استفاده از این کلاس، نیازی نیست که دانشجو نگران مقوله اول باشد و پس از تایپ مطالب خود می‌تواند آنها را با لاتک و ابزار آن اجرا کند تا پایان‌نامه خود را با قالب دانشگاه داشته باشد. همچنین با خواندن این راهنما از ملزومات محتوایی هر فصل پایان‌نامه نیز مطلع خواهد شد.

در ادامهٔ  مقدمهٔ این راهنما، ابتدا چگونگی استفاده از کلاس پایان‌نامه و فایل‌های همراه آن را به صورت فنی شرح می‌دهیم و سپس مطالبی را در مورد ویژگی‌های محتوایی فصل ۱ پایان‌نامه (یعنی مقدمه) خواهیم آورد.
بقیهٔ فصل‌های این راهنما، تنها خصوصیات محتوایی فصول مختلف پایان‌نامه را شرح خواهند داد. نهایتاً جهت یادآوری، در پیوست‌ها مطالبی دربارهٔ آشنایی با دستورات لاتک، مدیریت مراجع در لاتک و چگونگی رسم جداول، نمودارها و الگوریتم‌ها آورده خواهند شد.

\section{چگونگی استفاده از کلاس پایان‌نامه}
کلیه فایل‌های لازم برای حروف‌چینی با کلاس فوق، داخل پوشه‌ای به نام
\lr{tehran-thesis}
قرار داده شده است. توجه داشته باشید که برای استفاده از این کلاس باید فونت‌های
\lr{IRLotusICEE}
و
\lr{IRTitr}
را داشته باشید (که همراه با این کلاس هست و نیاز به نصب نیست).
قلم‌های
\lr{IRLotusICEE}
مستخرج از قلم‌های
\lr{IRLotus}
شورای عالی اطلاع‌رسانی هستند که توسط دکتر بابایی‌زاده اصلاحاتی روی آنها صورت پذیرفته است: تبدیل صفر توپر به صفر توخالی (جهت تمایز بیشتر با نقطه) و اضافه شدن
\textit{\textbf{حالت توپر و ایرانیک توأم}}،
که این موارد در قلم‌های شورای عالی اطلاع‌رسانی وجود ندارد.

\subsection{این همه فایل؟!}
\label{muchFiles}
از آنجایی که یک پایان‌نامه یا رساله، یک نوشته بلند محسوب می‌شود، لذا اگر همه تنظیمات و مطالب پایان‌نامه را داخل یک فایل قرار بدهیم، باعث شلوغی و سردرگمی می‌شود. به همین خاطر، قسمت‌های مختلف پایان‌نامه یا رساله  داخل فایل‌های جداگانه قرار گرفته است. مثلاً تنظیمات پایه‌ای کلاس داخل فایل
\lr{tehran-thesis.cls}، 
قسمت مشخصات فارسی پایان‌نامه داخل 
\lr{faTitle.tex}،
مطالب فصل اول داخل 
\lr{chapter1.tex}
و تنظیمات قابل تغییر توسط کاربر داخل 
\lr{commands.tex}،
قرار داده شده است.
\textbf{
	فایل اصلی این مجموعه، فایل
	\lr{main.tex}
	می‌باشد.
}
% یعنی بعد از تغییر فایل‌های دیگر، برای دیدن نتیجه تغییرات، باید این فایل را اجرا کرد. بقیه فایل‌ها به این فایل، کمک می‌کنند تا بتوانیم خروجی کار را ببینیم.
اگر به فایل 
\lr{main.tex}
دقت کنید، متوجه می‌شوید که قسمت‌های مختلف پایان‌نامه، توسط دستورهایی مانند 
\lr{input}
و
\lr{include}
به فایل اصلی، یعنی 
\lr{main.tex}
معرفی شده‌اند.
با توجه به ساختار محتوایی دستورالعمل، در فایل
\lr{main.tex}
فرض شده که پایان‌نامه یا رساله شما، از ۵ فصل و تعدادی پیوست تشکیل شده است. با اینحال، شما می‌توانید به راحتی فصل‌ها و پیوست‌ها را با صلاحدید اساتید راهنما، کم و زیاد کنید. این کار، بسیار ساده است. فرض کنید بخواهید یک فصل دیگر هم به پایان‌نامه اضافه کنید. برای این کار، کافی است یک فایل با نام دلخواه مثلاً 
\lr{chapter6}
و با پسوند 
\lr{.tex}
بسازید و آن را داخل پوشه 
\lr{tehran-thesis}
قرار دهید و سپس این فایل را با دستور 
\verb!\include{chapter6}!
داخل فایل
\lr{main.tex}
 فراخوانی کنید.

\subsection{از کجا شروع کنم؟}
قبل از هر چیز، باید یک توزیع تِک مناسب مانند تک‌لایو
\lr{(TeXLive)}
را روی سیستم خود نصب کنید. تک‌لایو  را می‌توانید از 
 \href{http://www.tug.org/texlive}{سایت رسمی آن}%
\LTRfootnote{\lr{\url{http://www.tug.org/texlive}}}
 دانلود کنید یا مستقیماً از مخازن توزیع لینوکس خود بگیرید (مثلاً در اوبونتو با دستور
\LRE{\verb!sudo apt install texlive-full!}).
برای نصب تک‌لایو و اجرای اسناد زی‌پرشین می‌توانید از
\href{http://parsilatex.com/site/shop/}{دی‌وی‌دی مجموعه پارسی‌لاتک}%
\LTRfootnote{\lr{\url{http://parsilatex.com/site/shop/}}}
و فایل راهنمای موجود در آن هم کمک بگیرید.

برای تایپ و پردازش اسناد لاتک باید از یک ویرایشگر مناسب استفاده کنید. ویرایشگرهای
\lr{TeXWroks},
\lr{TeXstudio},
\lr{Texmaker}
و
\lr{BiDiTeXmaker}
بدین منظور تولید شده‌اند. می‌توان ویرایش‌گر 
 \href{https://bitbucket.org/srazi/biditexmaker3}{\lr{BiDiTeXmaker}}%
 \LTRfootnote{\lr{\url{https://bitbucket.org/srazi/biditexmaker3}}}
را که بویژه برای کار با زی‌پرشین و مطالب دوجهته بهبود یافته است، بهینه‌ترین ویرایشگر لاتک برای کار با اسناد فارسی عنوان کرد.
 
حال اگر نوشتن \پ اولین تجربه شما از کار با لاتک است، توصیه می‌شود که یک‌بار به صورت اجمالی، کتاب «%
\href{http://www.tug.ctan.org/tex-archive/info/lshort/persian/lshort.pdf}{مقدمه‌ای نه چندان کوتاه بر
\lr{\LaTeXe}}%
\LTRfootnote{\lr{\url{http://www.tug.ctan.org/tex-archive/info/lshort/persian/lshort.pdf}\hfill}}»
ترجمه دکتر مهدی امیدعلی را مطالعه کنید. این کتاب، کتاب بسیار کاملی است که خیلی از نیازهای شما در ارتباط با حروف‌چینی را برطرف می‌کند.
اگر تک لایو کامل را داشته باشید، این کتاب را هم دارید. کافیست در خط فرمان دستور زیر را بزنید:
\begin{latin}
	\texttt{texdoc lshort-persian}
\end{latin}
اگر عجله دارید، برخی دستورات پایه‌ای مورد نیاز در پیوست \ref{App:latexIntro} بیان شده‌اند.
 
بعد از موارد گفته شده، فایل 
\lr{main.tex}
و
\lr{faTitle.tex}
را باز کنید و مشخصات پایان‌نامه خود مثل نام، نام خانوادگی، عنوان پایان‌نامه و ... را جایگزین مشخصات موجود در فایل
\lr{faTitle.tex}
 کنید. نیازی نیست نگران چینش این مشخصات در فایل پی‌دی‌اف خروجی باشید، زیرا کلاس 
\lr{tehran-thesis}
همه این کارها را بطور خودکار برای شما انجام می‌دهد. در ضمن، موقع تغییر دادن دستورهای داخل فایل
\lr{faTitle.tex}
 کاملاً دقت کنید؛ این دستورها، خیلی حساس هستند و ممکن است با یک تغییر کوچک، موقع اجرا، خطا بگیرید. برای دیدن خروجی کار، فایل 
\lr{faTitle.tex}
 را 
\lr{Save}
(نه 
\lr{Save As})
کنید و بعد به فایل 
\lr{main.tex}
برگشته و آن را اجرا کنید%
\footnote{
	البته فایلهای این مجموعه به گونه‌ای هستند که در
	\lr{TeXWorks} یا
	\lr{TeXstudio}
	بدون بازگشت به فایل اصلی، می‌توانید سند خود را اجرا کنید.
}.
 حال اگر می‌خواهید مشخصات انگلیسی \پ را هم عوض کنید، فایل 
\lr{enTitle.tex}
را باز کنید و مشخصات داخلش را تغییر دهید.
%\RTLfootnote{
%برای نوشتن پروژه کارشناسی، نیازی به وارد کردن مشخصات انگلیسی پروژه نیست. بنابراین، این مشخصات بطور خودکار، نادیده گرفته می‌شود.
%}
در اینجا هم برای دیدن خروجی باید این فایل را ذخیره کرده، بعد به فایل 
\lr{main.tex}
برگشته و آن را اجرا کرد.

برای راحتی بیشتر، کلاس 
\lr{tehran-thesis.cls}
طوری طراحی شده است که کافی است فقط  یک‌بار مشخصات \پ را (در فایل‌های
\lr{faTitle.tex}
و
\lr{enTitle.tex})
وارد کنید و هر جای دیگر که این مشخصات لازم باشند، به طور خودکار درج می‌شوند. با این حال، اگر مایل بودید، می‌توانید تنظیمات موجود را تغییر دهید؛ گرچه، در صورتیکه کاربر مبتدی هستید و یا با ساختار فایل‌های  
\lr{cls}
 آشنایی ندارید، بهتر است به فایل 
\lr{tehran-thesis.cls}
دست نزنید.

نکته دیگری که باید به آن توجه کنید این است که در قالب آماده شده، سه گزینه به نام‌های
\lr{bsc}،
\lr{msc}
و
\lr{phd}
برای نوشتن پروژه، پایان‌نامه و رساله، در نظر گرفته شده است. بنابراین اگر قصد تایپ پروژهٔ کارشناسی، پایان‌نامهٔ کارشناسی ارشد یا رسالهٔ دکتری را دارید، به ترتیب باید از گزینه‌های
\lr{bsc}،
\lr{msc}
و
\lr{phd}
در فایل 
\lr{main.tex}
استفاده کنید. با انتخاب هر کدام از این گزینه‌ها، تنظیمات مربوط به آنها به طور خودکار، اعمال می‌شود.


\subsection[مطالب پایان‌نامه را چطور بنویسم؟]
{مطالب \پ را چطور بنویسم؟}
\subsubsection{نوشتن فصل‌ها}
همان‌طور که در بخش \ref{muchFiles} گفته شد برای جلوگیری از شلوغی، قسمت‌های مختلف \پ از جمله فصل‌ها، در فایل‌های جداگانه‌ای قرار داده شده‌اند. 
مثلاً اگر می‌خواهید مطالب فصل ۱ را تایپ کنید، باید فایل‌های 
\lr{main.tex}
و
\lr{chapter1.tex}
را باز کرده و مطالب خود را جایگزین محتویات داخل 
\lr{chapter1.tex}
نمایید. دقت شود که در ابتدای برخی فایلها دستوراتی نوشته شده است و از شما خواسته شده که آن دستورات را حذف نکنید.

%توجه کنید که همان‌طور که قبلاً هم گفته شد، تنها فایل قابل اجرا، 
%\lr{main.tex}
%است. لذا برای دیدن حاصل (خروجی) فایل خود، باید  
%\lr{chapter1.tex}
%را ذخیره کرده و سپس فایل 
%\lr{main.tex}
%را اجرا کنید.

نکته بسیار مهمی که در اینجا باید گفته شود این است که سیستم \lr{\TeX}، محتویات یک فایل تِک را به ترتیب پردازش می‌کند.  بنابراین، اگر مثلاً  دو فصل اول خود را نوشته و خروجی آنها را دیده‌اید و مشغول تایپ مطالب فصل ۳ هستید، بهتر است
که دو دستور 
\verb!% !TeX root=main.tex
% دستور زیر باید در اولین فصل شما باشد. آن را حذف نکنید!
\pagenumbering{arabic}

\chapter{مقدمه}
\thispagestyle{empty}
\section{آشنایی با این راهنما}
حروف‌چینی پروژه کارشناسی، پایان‌نامه یا رساله یکی از موارد پرکاربرد استفاده از
\lr{\LaTeX}
و زی‌پرشین
\cite{Khalighi87xepersian}
است. یک پروژه، پایان‌نامه یا رساله، احتیاج به تنظیمات زیادی از نظر صفحه‌آرایی دارد که وقت زیادی از دانشجو می‌گیرد. به دلیل قابلیت‌های بسیار لاتک در حروف‌چینی، کلاسی با نام 
\lr{tehran-thesis}
برای حروف‌چینی پروژه‌ها، پایان‌نامه‌ها و رساله‌های دانشگاه تهران، بر مبنای کلاس مشابه
\lr{IUST-Thesis}
تهیه شده است. این کلاس و فایل‌های همراه آن به گونه‌ای طراحی شده است که مطابق با دستورالعمل نگارش و تدوین پایان‌نامه کارشناسی ارشد و دکتری پردیس دانشکده‌های فنی دانشگاه تهران
\cite{UTThesisGuide}
باشد.

دستورالعمل نگارش و تدوین پایان‌نامه دانشگاه تهران به دو مقوله می‌پردازد، اول قالب و چگونگی صفحه‌آرایی پایان‌نامه، مانند اندازه و نوع قلم بخشهای مختلف، چینش فصلها، قالب مراجع و مواردی از این قبیل و دوم محتوای هر فصل پایان‌نامه. 
درصورت استفاده از این کلاس، نیازی نیست که دانشجو نگران مقوله اول باشد و پس از تایپ مطالب خود می‌تواند آنها را با لاتک و ابزار آن اجرا کند تا پایان‌نامه خود را با قالب دانشگاه داشته باشد. همچنین با خواندن این راهنما از ملزومات محتوایی هر فصل پایان‌نامه نیز مطلع خواهد شد.

در ادامهٔ  مقدمهٔ این راهنما، ابتدا چگونگی استفاده از کلاس پایان‌نامه و فایل‌های همراه آن را به صورت فنی شرح می‌دهیم و سپس مطالبی را در مورد ویژگی‌های محتوایی فصل ۱ پایان‌نامه (یعنی مقدمه) خواهیم آورد.
بقیهٔ فصل‌های این راهنما، تنها خصوصیات محتوایی فصول مختلف پایان‌نامه را شرح خواهند داد. نهایتاً جهت یادآوری، در پیوست‌ها مطالبی دربارهٔ آشنایی با دستورات لاتک، مدیریت مراجع در لاتک و چگونگی رسم جداول، نمودارها و الگوریتم‌ها آورده خواهند شد.

\section{چگونگی استفاده از کلاس پایان‌نامه}
کلیه فایل‌های لازم برای حروف‌چینی با کلاس فوق، داخل پوشه‌ای به نام
\lr{tehran-thesis}
قرار داده شده است. توجه داشته باشید که برای استفاده از این کلاس باید فونت‌های
\lr{IRLotusICEE}
و
\lr{IRTitr}
را داشته باشید (که همراه با این کلاس هست و نیاز به نصب نیست).
قلم‌های
\lr{IRLotusICEE}
مستخرج از قلم‌های
\lr{IRLotus}
شورای عالی اطلاع‌رسانی هستند که توسط دکتر بابایی‌زاده اصلاحاتی روی آنها صورت پذیرفته است: تبدیل صفر توپر به صفر توخالی (جهت تمایز بیشتر با نقطه) و اضافه شدن
\textit{\textbf{حالت توپر و ایرانیک توأم}}،
که این موارد در قلم‌های شورای عالی اطلاع‌رسانی وجود ندارد.

\subsection{این همه فایل؟!}
\label{muchFiles}
از آنجایی که یک پایان‌نامه یا رساله، یک نوشته بلند محسوب می‌شود، لذا اگر همه تنظیمات و مطالب پایان‌نامه را داخل یک فایل قرار بدهیم، باعث شلوغی و سردرگمی می‌شود. به همین خاطر، قسمت‌های مختلف پایان‌نامه یا رساله  داخل فایل‌های جداگانه قرار گرفته است. مثلاً تنظیمات پایه‌ای کلاس داخل فایل
\lr{tehran-thesis.cls}، 
قسمت مشخصات فارسی پایان‌نامه داخل 
\lr{faTitle.tex}،
مطالب فصل اول داخل 
\lr{chapter1.tex}
و تنظیمات قابل تغییر توسط کاربر داخل 
\lr{commands.tex}،
قرار داده شده است.
\textbf{
	فایل اصلی این مجموعه، فایل
	\lr{main.tex}
	می‌باشد.
}
% یعنی بعد از تغییر فایل‌های دیگر، برای دیدن نتیجه تغییرات، باید این فایل را اجرا کرد. بقیه فایل‌ها به این فایل، کمک می‌کنند تا بتوانیم خروجی کار را ببینیم.
اگر به فایل 
\lr{main.tex}
دقت کنید، متوجه می‌شوید که قسمت‌های مختلف پایان‌نامه، توسط دستورهایی مانند 
\lr{input}
و
\lr{include}
به فایل اصلی، یعنی 
\lr{main.tex}
معرفی شده‌اند.
با توجه به ساختار محتوایی دستورالعمل، در فایل
\lr{main.tex}
فرض شده که پایان‌نامه یا رساله شما، از ۵ فصل و تعدادی پیوست تشکیل شده است. با اینحال، شما می‌توانید به راحتی فصل‌ها و پیوست‌ها را با صلاحدید اساتید راهنما، کم و زیاد کنید. این کار، بسیار ساده است. فرض کنید بخواهید یک فصل دیگر هم به پایان‌نامه اضافه کنید. برای این کار، کافی است یک فایل با نام دلخواه مثلاً 
\lr{chapter6}
و با پسوند 
\lr{.tex}
بسازید و آن را داخل پوشه 
\lr{tehran-thesis}
قرار دهید و سپس این فایل را با دستور 
\verb!\include{chapter6}!
داخل فایل
\lr{main.tex}
 فراخوانی کنید.

\subsection{از کجا شروع کنم؟}
قبل از هر چیز، باید یک توزیع تِک مناسب مانند تک‌لایو
\lr{(TeXLive)}
را روی سیستم خود نصب کنید. تک‌لایو  را می‌توانید از 
 \href{http://www.tug.org/texlive}{سایت رسمی آن}%
\LTRfootnote{\lr{\url{http://www.tug.org/texlive}}}
 دانلود کنید یا مستقیماً از مخازن توزیع لینوکس خود بگیرید (مثلاً در اوبونتو با دستور
\LRE{\verb!sudo apt install texlive-full!}).
برای نصب تک‌لایو و اجرای اسناد زی‌پرشین می‌توانید از
\href{http://parsilatex.com/site/shop/}{دی‌وی‌دی مجموعه پارسی‌لاتک}%
\LTRfootnote{\lr{\url{http://parsilatex.com/site/shop/}}}
و فایل راهنمای موجود در آن هم کمک بگیرید.

برای تایپ و پردازش اسناد لاتک باید از یک ویرایشگر مناسب استفاده کنید. ویرایشگرهای
\lr{TeXWroks},
\lr{TeXstudio},
\lr{Texmaker}
و
\lr{BiDiTeXmaker}
بدین منظور تولید شده‌اند. می‌توان ویرایش‌گر 
 \href{https://bitbucket.org/srazi/biditexmaker3}{\lr{BiDiTeXmaker}}%
 \LTRfootnote{\lr{\url{https://bitbucket.org/srazi/biditexmaker3}}}
را که بویژه برای کار با زی‌پرشین و مطالب دوجهته بهبود یافته است، بهینه‌ترین ویرایشگر لاتک برای کار با اسناد فارسی عنوان کرد.
 
حال اگر نوشتن \پ اولین تجربه شما از کار با لاتک است، توصیه می‌شود که یک‌بار به صورت اجمالی، کتاب «%
\href{http://www.tug.ctan.org/tex-archive/info/lshort/persian/lshort.pdf}{مقدمه‌ای نه چندان کوتاه بر
\lr{\LaTeXe}}%
\LTRfootnote{\lr{\url{http://www.tug.ctan.org/tex-archive/info/lshort/persian/lshort.pdf}\hfill}}»
ترجمه دکتر مهدی امیدعلی را مطالعه کنید. این کتاب، کتاب بسیار کاملی است که خیلی از نیازهای شما در ارتباط با حروف‌چینی را برطرف می‌کند.
اگر تک لایو کامل را داشته باشید، این کتاب را هم دارید. کافیست در خط فرمان دستور زیر را بزنید:
\begin{latin}
	\texttt{texdoc lshort-persian}
\end{latin}
اگر عجله دارید، برخی دستورات پایه‌ای مورد نیاز در پیوست \ref{App:latexIntro} بیان شده‌اند.
 
بعد از موارد گفته شده، فایل 
\lr{main.tex}
و
\lr{faTitle.tex}
را باز کنید و مشخصات پایان‌نامه خود مثل نام، نام خانوادگی، عنوان پایان‌نامه و ... را جایگزین مشخصات موجود در فایل
\lr{faTitle.tex}
 کنید. نیازی نیست نگران چینش این مشخصات در فایل پی‌دی‌اف خروجی باشید، زیرا کلاس 
\lr{tehran-thesis}
همه این کارها را بطور خودکار برای شما انجام می‌دهد. در ضمن، موقع تغییر دادن دستورهای داخل فایل
\lr{faTitle.tex}
 کاملاً دقت کنید؛ این دستورها، خیلی حساس هستند و ممکن است با یک تغییر کوچک، موقع اجرا، خطا بگیرید. برای دیدن خروجی کار، فایل 
\lr{faTitle.tex}
 را 
\lr{Save}
(نه 
\lr{Save As})
کنید و بعد به فایل 
\lr{main.tex}
برگشته و آن را اجرا کنید%
\footnote{
	البته فایلهای این مجموعه به گونه‌ای هستند که در
	\lr{TeXWorks} یا
	\lr{TeXstudio}
	بدون بازگشت به فایل اصلی، می‌توانید سند خود را اجرا کنید.
}.
 حال اگر می‌خواهید مشخصات انگلیسی \پ را هم عوض کنید، فایل 
\lr{enTitle.tex}
را باز کنید و مشخصات داخلش را تغییر دهید.
%\RTLfootnote{
%برای نوشتن پروژه کارشناسی، نیازی به وارد کردن مشخصات انگلیسی پروژه نیست. بنابراین، این مشخصات بطور خودکار، نادیده گرفته می‌شود.
%}
در اینجا هم برای دیدن خروجی باید این فایل را ذخیره کرده، بعد به فایل 
\lr{main.tex}
برگشته و آن را اجرا کرد.

برای راحتی بیشتر، کلاس 
\lr{tehran-thesis.cls}
طوری طراحی شده است که کافی است فقط  یک‌بار مشخصات \پ را (در فایل‌های
\lr{faTitle.tex}
و
\lr{enTitle.tex})
وارد کنید و هر جای دیگر که این مشخصات لازم باشند، به طور خودکار درج می‌شوند. با این حال، اگر مایل بودید، می‌توانید تنظیمات موجود را تغییر دهید؛ گرچه، در صورتیکه کاربر مبتدی هستید و یا با ساختار فایل‌های  
\lr{cls}
 آشنایی ندارید، بهتر است به فایل 
\lr{tehran-thesis.cls}
دست نزنید.

نکته دیگری که باید به آن توجه کنید این است که در قالب آماده شده، سه گزینه به نام‌های
\lr{bsc}،
\lr{msc}
و
\lr{phd}
برای نوشتن پروژه، پایان‌نامه و رساله، در نظر گرفته شده است. بنابراین اگر قصد تایپ پروژهٔ کارشناسی، پایان‌نامهٔ کارشناسی ارشد یا رسالهٔ دکتری را دارید، به ترتیب باید از گزینه‌های
\lr{bsc}،
\lr{msc}
و
\lr{phd}
در فایل 
\lr{main.tex}
استفاده کنید. با انتخاب هر کدام از این گزینه‌ها، تنظیمات مربوط به آنها به طور خودکار، اعمال می‌شود.


\subsection[مطالب پایان‌نامه را چطور بنویسم؟]
{مطالب \پ را چطور بنویسم؟}
\subsubsection{نوشتن فصل‌ها}
همان‌طور که در بخش \ref{muchFiles} گفته شد برای جلوگیری از شلوغی، قسمت‌های مختلف \پ از جمله فصل‌ها، در فایل‌های جداگانه‌ای قرار داده شده‌اند. 
مثلاً اگر می‌خواهید مطالب فصل ۱ را تایپ کنید، باید فایل‌های 
\lr{main.tex}
و
\lr{chapter1.tex}
را باز کرده و مطالب خود را جایگزین محتویات داخل 
\lr{chapter1.tex}
نمایید. دقت شود که در ابتدای برخی فایلها دستوراتی نوشته شده است و از شما خواسته شده که آن دستورات را حذف نکنید.

%توجه کنید که همان‌طور که قبلاً هم گفته شد، تنها فایل قابل اجرا، 
%\lr{main.tex}
%است. لذا برای دیدن حاصل (خروجی) فایل خود، باید  
%\lr{chapter1.tex}
%را ذخیره کرده و سپس فایل 
%\lr{main.tex}
%را اجرا کنید.

نکته بسیار مهمی که در اینجا باید گفته شود این است که سیستم \lr{\TeX}، محتویات یک فایل تِک را به ترتیب پردازش می‌کند.  بنابراین، اگر مثلاً  دو فصل اول خود را نوشته و خروجی آنها را دیده‌اید و مشغول تایپ مطالب فصل ۳ هستید، بهتر است
که دو دستور 
\verb!% !TeX root=main.tex
% دستور زیر باید در اولین فصل شما باشد. آن را حذف نکنید!
\pagenumbering{arabic}

\chapter{مقدمه}
\thispagestyle{empty}
\section{آشنایی با این راهنما}
حروف‌چینی پروژه کارشناسی، پایان‌نامه یا رساله یکی از موارد پرکاربرد استفاده از
\lr{\LaTeX}
و زی‌پرشین
\cite{Khalighi87xepersian}
است. یک پروژه، پایان‌نامه یا رساله، احتیاج به تنظیمات زیادی از نظر صفحه‌آرایی دارد که وقت زیادی از دانشجو می‌گیرد. به دلیل قابلیت‌های بسیار لاتک در حروف‌چینی، کلاسی با نام 
\lr{tehran-thesis}
برای حروف‌چینی پروژه‌ها، پایان‌نامه‌ها و رساله‌های دانشگاه تهران، بر مبنای کلاس مشابه
\lr{IUST-Thesis}
تهیه شده است. این کلاس و فایل‌های همراه آن به گونه‌ای طراحی شده است که مطابق با دستورالعمل نگارش و تدوین پایان‌نامه کارشناسی ارشد و دکتری پردیس دانشکده‌های فنی دانشگاه تهران
\cite{UTThesisGuide}
باشد.

دستورالعمل نگارش و تدوین پایان‌نامه دانشگاه تهران به دو مقوله می‌پردازد، اول قالب و چگونگی صفحه‌آرایی پایان‌نامه، مانند اندازه و نوع قلم بخشهای مختلف، چینش فصلها، قالب مراجع و مواردی از این قبیل و دوم محتوای هر فصل پایان‌نامه. 
درصورت استفاده از این کلاس، نیازی نیست که دانشجو نگران مقوله اول باشد و پس از تایپ مطالب خود می‌تواند آنها را با لاتک و ابزار آن اجرا کند تا پایان‌نامه خود را با قالب دانشگاه داشته باشد. همچنین با خواندن این راهنما از ملزومات محتوایی هر فصل پایان‌نامه نیز مطلع خواهد شد.

در ادامهٔ  مقدمهٔ این راهنما، ابتدا چگونگی استفاده از کلاس پایان‌نامه و فایل‌های همراه آن را به صورت فنی شرح می‌دهیم و سپس مطالبی را در مورد ویژگی‌های محتوایی فصل ۱ پایان‌نامه (یعنی مقدمه) خواهیم آورد.
بقیهٔ فصل‌های این راهنما، تنها خصوصیات محتوایی فصول مختلف پایان‌نامه را شرح خواهند داد. نهایتاً جهت یادآوری، در پیوست‌ها مطالبی دربارهٔ آشنایی با دستورات لاتک، مدیریت مراجع در لاتک و چگونگی رسم جداول، نمودارها و الگوریتم‌ها آورده خواهند شد.

\section{چگونگی استفاده از کلاس پایان‌نامه}
کلیه فایل‌های لازم برای حروف‌چینی با کلاس فوق، داخل پوشه‌ای به نام
\lr{tehran-thesis}
قرار داده شده است. توجه داشته باشید که برای استفاده از این کلاس باید فونت‌های
\lr{IRLotusICEE}
و
\lr{IRTitr}
را داشته باشید (که همراه با این کلاس هست و نیاز به نصب نیست).
قلم‌های
\lr{IRLotusICEE}
مستخرج از قلم‌های
\lr{IRLotus}
شورای عالی اطلاع‌رسانی هستند که توسط دکتر بابایی‌زاده اصلاحاتی روی آنها صورت پذیرفته است: تبدیل صفر توپر به صفر توخالی (جهت تمایز بیشتر با نقطه) و اضافه شدن
\textit{\textbf{حالت توپر و ایرانیک توأم}}،
که این موارد در قلم‌های شورای عالی اطلاع‌رسانی وجود ندارد.

\subsection{این همه فایل؟!}
\label{muchFiles}
از آنجایی که یک پایان‌نامه یا رساله، یک نوشته بلند محسوب می‌شود، لذا اگر همه تنظیمات و مطالب پایان‌نامه را داخل یک فایل قرار بدهیم، باعث شلوغی و سردرگمی می‌شود. به همین خاطر، قسمت‌های مختلف پایان‌نامه یا رساله  داخل فایل‌های جداگانه قرار گرفته است. مثلاً تنظیمات پایه‌ای کلاس داخل فایل
\lr{tehran-thesis.cls}، 
قسمت مشخصات فارسی پایان‌نامه داخل 
\lr{faTitle.tex}،
مطالب فصل اول داخل 
\lr{chapter1.tex}
و تنظیمات قابل تغییر توسط کاربر داخل 
\lr{commands.tex}،
قرار داده شده است.
\textbf{
	فایل اصلی این مجموعه، فایل
	\lr{main.tex}
	می‌باشد.
}
% یعنی بعد از تغییر فایل‌های دیگر، برای دیدن نتیجه تغییرات، باید این فایل را اجرا کرد. بقیه فایل‌ها به این فایل، کمک می‌کنند تا بتوانیم خروجی کار را ببینیم.
اگر به فایل 
\lr{main.tex}
دقت کنید، متوجه می‌شوید که قسمت‌های مختلف پایان‌نامه، توسط دستورهایی مانند 
\lr{input}
و
\lr{include}
به فایل اصلی، یعنی 
\lr{main.tex}
معرفی شده‌اند.
با توجه به ساختار محتوایی دستورالعمل، در فایل
\lr{main.tex}
فرض شده که پایان‌نامه یا رساله شما، از ۵ فصل و تعدادی پیوست تشکیل شده است. با اینحال، شما می‌توانید به راحتی فصل‌ها و پیوست‌ها را با صلاحدید اساتید راهنما، کم و زیاد کنید. این کار، بسیار ساده است. فرض کنید بخواهید یک فصل دیگر هم به پایان‌نامه اضافه کنید. برای این کار، کافی است یک فایل با نام دلخواه مثلاً 
\lr{chapter6}
و با پسوند 
\lr{.tex}
بسازید و آن را داخل پوشه 
\lr{tehran-thesis}
قرار دهید و سپس این فایل را با دستور 
\verb!\include{chapter6}!
داخل فایل
\lr{main.tex}
 فراخوانی کنید.

\subsection{از کجا شروع کنم؟}
قبل از هر چیز، باید یک توزیع تِک مناسب مانند تک‌لایو
\lr{(TeXLive)}
را روی سیستم خود نصب کنید. تک‌لایو  را می‌توانید از 
 \href{http://www.tug.org/texlive}{سایت رسمی آن}%
\LTRfootnote{\lr{\url{http://www.tug.org/texlive}}}
 دانلود کنید یا مستقیماً از مخازن توزیع لینوکس خود بگیرید (مثلاً در اوبونتو با دستور
\LRE{\verb!sudo apt install texlive-full!}).
برای نصب تک‌لایو و اجرای اسناد زی‌پرشین می‌توانید از
\href{http://parsilatex.com/site/shop/}{دی‌وی‌دی مجموعه پارسی‌لاتک}%
\LTRfootnote{\lr{\url{http://parsilatex.com/site/shop/}}}
و فایل راهنمای موجود در آن هم کمک بگیرید.

برای تایپ و پردازش اسناد لاتک باید از یک ویرایشگر مناسب استفاده کنید. ویرایشگرهای
\lr{TeXWroks},
\lr{TeXstudio},
\lr{Texmaker}
و
\lr{BiDiTeXmaker}
بدین منظور تولید شده‌اند. می‌توان ویرایش‌گر 
 \href{https://bitbucket.org/srazi/biditexmaker3}{\lr{BiDiTeXmaker}}%
 \LTRfootnote{\lr{\url{https://bitbucket.org/srazi/biditexmaker3}}}
را که بویژه برای کار با زی‌پرشین و مطالب دوجهته بهبود یافته است، بهینه‌ترین ویرایشگر لاتک برای کار با اسناد فارسی عنوان کرد.
 
حال اگر نوشتن \پ اولین تجربه شما از کار با لاتک است، توصیه می‌شود که یک‌بار به صورت اجمالی، کتاب «%
\href{http://www.tug.ctan.org/tex-archive/info/lshort/persian/lshort.pdf}{مقدمه‌ای نه چندان کوتاه بر
\lr{\LaTeXe}}%
\LTRfootnote{\lr{\url{http://www.tug.ctan.org/tex-archive/info/lshort/persian/lshort.pdf}\hfill}}»
ترجمه دکتر مهدی امیدعلی را مطالعه کنید. این کتاب، کتاب بسیار کاملی است که خیلی از نیازهای شما در ارتباط با حروف‌چینی را برطرف می‌کند.
اگر تک لایو کامل را داشته باشید، این کتاب را هم دارید. کافیست در خط فرمان دستور زیر را بزنید:
\begin{latin}
	\texttt{texdoc lshort-persian}
\end{latin}
اگر عجله دارید، برخی دستورات پایه‌ای مورد نیاز در پیوست \ref{App:latexIntro} بیان شده‌اند.
 
بعد از موارد گفته شده، فایل 
\lr{main.tex}
و
\lr{faTitle.tex}
را باز کنید و مشخصات پایان‌نامه خود مثل نام، نام خانوادگی، عنوان پایان‌نامه و ... را جایگزین مشخصات موجود در فایل
\lr{faTitle.tex}
 کنید. نیازی نیست نگران چینش این مشخصات در فایل پی‌دی‌اف خروجی باشید، زیرا کلاس 
\lr{tehran-thesis}
همه این کارها را بطور خودکار برای شما انجام می‌دهد. در ضمن، موقع تغییر دادن دستورهای داخل فایل
\lr{faTitle.tex}
 کاملاً دقت کنید؛ این دستورها، خیلی حساس هستند و ممکن است با یک تغییر کوچک، موقع اجرا، خطا بگیرید. برای دیدن خروجی کار، فایل 
\lr{faTitle.tex}
 را 
\lr{Save}
(نه 
\lr{Save As})
کنید و بعد به فایل 
\lr{main.tex}
برگشته و آن را اجرا کنید%
\footnote{
	البته فایلهای این مجموعه به گونه‌ای هستند که در
	\lr{TeXWorks} یا
	\lr{TeXstudio}
	بدون بازگشت به فایل اصلی، می‌توانید سند خود را اجرا کنید.
}.
 حال اگر می‌خواهید مشخصات انگلیسی \پ را هم عوض کنید، فایل 
\lr{enTitle.tex}
را باز کنید و مشخصات داخلش را تغییر دهید.
%\RTLfootnote{
%برای نوشتن پروژه کارشناسی، نیازی به وارد کردن مشخصات انگلیسی پروژه نیست. بنابراین، این مشخصات بطور خودکار، نادیده گرفته می‌شود.
%}
در اینجا هم برای دیدن خروجی باید این فایل را ذخیره کرده، بعد به فایل 
\lr{main.tex}
برگشته و آن را اجرا کرد.

برای راحتی بیشتر، کلاس 
\lr{tehran-thesis.cls}
طوری طراحی شده است که کافی است فقط  یک‌بار مشخصات \پ را (در فایل‌های
\lr{faTitle.tex}
و
\lr{enTitle.tex})
وارد کنید و هر جای دیگر که این مشخصات لازم باشند، به طور خودکار درج می‌شوند. با این حال، اگر مایل بودید، می‌توانید تنظیمات موجود را تغییر دهید؛ گرچه، در صورتیکه کاربر مبتدی هستید و یا با ساختار فایل‌های  
\lr{cls}
 آشنایی ندارید، بهتر است به فایل 
\lr{tehran-thesis.cls}
دست نزنید.

نکته دیگری که باید به آن توجه کنید این است که در قالب آماده شده، سه گزینه به نام‌های
\lr{bsc}،
\lr{msc}
و
\lr{phd}
برای نوشتن پروژه، پایان‌نامه و رساله، در نظر گرفته شده است. بنابراین اگر قصد تایپ پروژهٔ کارشناسی، پایان‌نامهٔ کارشناسی ارشد یا رسالهٔ دکتری را دارید، به ترتیب باید از گزینه‌های
\lr{bsc}،
\lr{msc}
و
\lr{phd}
در فایل 
\lr{main.tex}
استفاده کنید. با انتخاب هر کدام از این گزینه‌ها، تنظیمات مربوط به آنها به طور خودکار، اعمال می‌شود.


\subsection[مطالب پایان‌نامه را چطور بنویسم؟]
{مطالب \پ را چطور بنویسم؟}
\subsubsection{نوشتن فصل‌ها}
همان‌طور که در بخش \ref{muchFiles} گفته شد برای جلوگیری از شلوغی، قسمت‌های مختلف \پ از جمله فصل‌ها، در فایل‌های جداگانه‌ای قرار داده شده‌اند. 
مثلاً اگر می‌خواهید مطالب فصل ۱ را تایپ کنید، باید فایل‌های 
\lr{main.tex}
و
\lr{chapter1.tex}
را باز کرده و مطالب خود را جایگزین محتویات داخل 
\lr{chapter1.tex}
نمایید. دقت شود که در ابتدای برخی فایلها دستوراتی نوشته شده است و از شما خواسته شده که آن دستورات را حذف نکنید.

%توجه کنید که همان‌طور که قبلاً هم گفته شد، تنها فایل قابل اجرا، 
%\lr{main.tex}
%است. لذا برای دیدن حاصل (خروجی) فایل خود، باید  
%\lr{chapter1.tex}
%را ذخیره کرده و سپس فایل 
%\lr{main.tex}
%را اجرا کنید.

نکته بسیار مهمی که در اینجا باید گفته شود این است که سیستم \lr{\TeX}، محتویات یک فایل تِک را به ترتیب پردازش می‌کند.  بنابراین، اگر مثلاً  دو فصل اول خود را نوشته و خروجی آنها را دیده‌اید و مشغول تایپ مطالب فصل ۳ هستید، بهتر است
که دو دستور 
\verb!\include{chapter1}!
و
\verb!\include{chapter2}!
را در فایل 
\lr{main.tex}،
غیرفعال%
\footnote{
برای غیرفعال کردن یک دستور، کافی است در ابتدای آن، علامت درصد انگلیسی (\%) بگذارید.
}
 کنید. در غیر این صورت، ابتدا مطالب دو فصل اول پردازش شده و سپس مطالب فصل ۳ پردازش می‌شود که این کار باعث طولانی شدن زمان پردازش می‌گردد. هر زمان که خروجی کل \پ را خواستید، تمام فصل‌ها را دوباره در
\lr{main.tex}
فعال نمائید.
بدیهتاً لازم نیست فصل‌های \پ را به ترتیب تایپ کنید. مثلاً می‌توانید ابتدا مطالب فصل ۳ را تایپ نموده و سپس مطالب فصل ۱ را تایپ کنید. 
\subsubsection{مراجع}
برای وارد کردن مراجع \پ کافی است فایل 
\lr{MyReferences.bib}
را باز کرده و مراجع خود را به شکل اقلام نمونهٔ داخل آن، وارد کنید.  سپس از \lr{bibtex} برای تولید مراجع با قالب مناسب استفاده نمائید. برای توضیحات بیشتر بخش \ref{Sec:Ref} از پیوست \ref{App:latexIntro} و نیز پیوست \ref{App:RefMan} را ببینید.

\subsubsection{واژه‌نامه فارسی به انگلیسی و برعکس}
برای وارد کردن واژه‌نامه فارسی به انگلیسی و برعکس، چنانچه کاربر مبتدی هستید بهتر است مانند روش بکار رفته در فایل‌های 
\lr{dicfa2en.tex}
و
\lr{dicen2fa.tex}
عمل کنید. اما چنانچه کاربر پیشرفته هستید، بهتر است از بسته
\lr{glossaries}
استفاده کنید. راهنمای این بسته را می‌توانید به راحتی و با یک جستجوی ساده در اینترنت پیدا کنید.
\subsubsection{نمایه}
برای وارد کردن نمایه، باید از 
\lr{xindy}
استفاده کنید. 
%زیرا 
%\lr{MakeIndex}
%با حروف «گ»، «چ»، «پ»، «ژ» و «ک» مشکل دارد و ترتیب الفبایی این حروف را رعایت نمی‌کند. همچنین، فاصله بین هر گروه از کلمات در 
%\lr{MakeIndex}،
%به درستی رعایت نمی‌شود که باعث زشت شدن حروف‌چینی این قسمت می‌شود. 
راهنمای چگونگی کار با 
\lr{xindy} 
را می‌توانید در ویکی پارسی‌لاتک و یا مثالهای موجود در دی‌وی‌دی «مجموعه پارسی‌لاتک»، پیدا کنید.

\subsection{اگر سوالی داشتم، از کی بپرسم؟}
برای پرسیدن سوال‌های خود موقع حروف‌چینی با زی‌پرشین، می‌توانید به
\href{http://qa.parsilatex.com}{سایت پرسش و پاسخ پارسی‌لاتک}%
\LTRfootnote{http://qa.parsilatex.com}
یا
\href{http://forum.parsilatex.com}{بایگانی تالارگفتگوی قدیمی پارسی‌لاتک}%
\LTRfootnote{http://forum.parsilatex.com}
مراجعه کنید. شما هم می‌توانید روزی به سوال‌های دیگران در اینترنت جواب دهید.
بستهٔ زی‌پرشین و بسیاری از بسته‌های مرتبط با آن مانند
\lr{bidi} و
\lr{Persian-bib}،
مجموعه پارسی‌لاتک، مثالهای مختلف موجود در آن، قالب پایان‌نامه دانشگاههای مختلف و سایت پارسی‌لاتک همه به صورت داوطلبانه توسط افراد گروه پارسی‌لاتک و گروه
\lr{Persian TeX}
و بدون هیچ کمک مالی انجام شده‌اند. کار اصلی نوشتن و توسعه زی‌پرشین توسط آقای وفا خلیقی انجام شده است که این کار بزرگ را به انجام رساندند.
اگر مایل به کمک به گروه پارسی‌لاتک هستید به سایت این گروه مراجعه فرمایید:
\begin{center}
	\url{http://www.parsilatex.com}
\end{center}

\section{محتویات فصل اول یک پایان‌نامه}
فصل اول یک پایان‌نامه باید به مقدمه یا کلیات تحقیق بپردازد.
هدف از فصل مقدمه%
\LTRfootnote{Introduction}،
شرح مختصر مسأله تحقیق، اهمیت و انگیزه محقق از پرداختن به آن موضوع، بهمراه اشاره‌ای كوتاه به روش و مراحل تحقیق است. مقدمه، اولين فصل از ساختار اصلی \پ بوده و زمینه اطلاعاتی لازم را برای خواننده فراهم می‌آورد. در طول مقدمه باید سعی شود موضوع تحقیق با زبانی روشن، ساده و بطور عمیق و هدفمند به خواننده معرفی شود. این فصل باید خواننده را مجذوب و اهميت موضوع تحقيق را آشکار سازد. در مقدمه باید با ارائهٔ سوابق، شواهد تحقيقی و اطلاعات موجود (با ذکر منبع) با روشی منظم، منطقی و هدف‌دار، خواننده را جهت داد و به سوی راه حل مورد نظر هدايت کرد. مقدمه مناسب‌ترين جا برای ارائهٔ اختصارات و بعضی توضيحات کلی است، توضيحاتی که شايد نتوان در مباحث ديگر آنها را شرح داد.

مقدمه، یکی از ارکان اساسی و اصلی پایان نامه است که مهمترین قسمت‌های آن عبارتند از: 


\section{جمع‌بندی}!
و
\verb!% !TeX root=main.tex
\chapter{مروری بر مطالعات انجام شده}
%\thispagestyle{empty} 
\section{مقدمه}
هدف از اين فصل که با عنوان‌های  «مروری بر ادبیات موضوع%
\LTRfootnote{Literature Review}»،
«مروری بر منابع» و يا «مروری بر پیشینه تحقیق%
\LTRfootnote{Background Research}»
معرفی می‌شود، بررسی و طبقه‌بندی یافته‌های تحقیقات دیگر محققان در سطح دنیا و تعیین و شناسایی خلأهای تحقیقاتی است. آنچه را که تحقیق شما به دانش موجود اضافه می‌کند، مشخص کنید. طرح پیشینه تحقیق%
\LTRfootnote{Background Information}
یک مرور محققانه است و تا آنجا باید پیش برود که پیش‌زمینهٔ تاریخی مناسبی از تحقیق را بیان کند و جایگاه تحقیق فعلی را در میان آثار پیشین نشان دهد. برای این منظور منابع مرتبط با تحقیق را بررسی کنید، البته نه آنچنان گسترده که کل پیشینه تاریخی بحث را در برگیرد. برای نوشتن این بخش:
\begin{itemize}
	\item
	دانستنی‌های موجود و پیش‌زمینهٔ تاریخی و وضعیت کنونی موضوع را چنان بیان کنید که خواننده بدون مراجعه به منابع پیشین، نتایج حاصل از مطالعات قبلی را درک و ارزیابی کند.
	\item
	نشان دهید که بر موضوع احاطه دارید. پرسش تحقیق را همراه بحث و جدل‌ها و مسائل مطرح شده بیان کنید و مهم‌ترین تحقیق‌های انجام شده در این زمینه را معرفی نمائید.
	\item
	ابتدا مطالب عمومی‌تر و سپس پژوهش‌های مشابه با کار خود را معرفی کرده و نشان دهید که تحقیق شما از چه جنبه‌ای با کار دیگران تشابه یا تفاوت دارد.
	\item
	اگر کارهای قبلی را خلاصه کرده‌اید، از پرداختن به جزئیات غیرضروری بپرهیزید. در عوض، بر یافته‌ها و مسائل روش‌شناختی مرتبط و نتایج اصلی تأکید کنید و اگر بررسی‌ها و منابع مروری عمومی دربارهٔ موضوع موجود است، خواننده را به آنها ارجاع دهید.
\end{itemize}

\section{تعاریف، اصول و مبانی نظری}
این قسمت ارائهٔ خلاصه‌ای از دانش کلاسیک موضوع است. این بخش الزامی نیست و بستگی به نظر استاد راهنما دارد.

\section{مروری بر ادبیات موضوع}
در این قسمت باید به کارهای مشابه دیگران در گذشته اشاره کرد و وزن بیشتر این قسمت بهتر است به مقالات ژورنالی سال‌های اخیر (۲ تا ۳ سال) تخصیص داده شود. به نتایج کارهای دیگران با ذکر دقیق مراجع باید اشاره شده و جایگاه و تفاوت تحقیق شما نیز با کارهای دیگران مشخص شود. استفاده از مقالات ژورنال‌های معتبر در دو یا سه سال اخیر، می‌تواند به اعتبار کار شما بیافزاید.

\section{نتیجه‌گیری}
‌در نتیجه‌گیری آخر این فصل، با توجه به بررسی انجام شده بر روی مراجع تحقيق، بخش‌های قابل گسترش و تحقیق در آن حیطه و چشم‌اندازهای تحقیق مورد بررسی قرار می‌گیرند.	در برخی از تحقیقات، نتیجه نهایی فصل روش تحقیق، ارائهٔ یک چارچوب کار تحقیقی 
\lr{(research framework)}
است.!
را در فایل 
\lr{main.tex}،
غیرفعال%
\footnote{
برای غیرفعال کردن یک دستور، کافی است در ابتدای آن، علامت درصد انگلیسی (\%) بگذارید.
}
 کنید. در غیر این صورت، ابتدا مطالب دو فصل اول پردازش شده و سپس مطالب فصل ۳ پردازش می‌شود که این کار باعث طولانی شدن زمان پردازش می‌گردد. هر زمان که خروجی کل \پ را خواستید، تمام فصل‌ها را دوباره در
\lr{main.tex}
فعال نمائید.
بدیهتاً لازم نیست فصل‌های \پ را به ترتیب تایپ کنید. مثلاً می‌توانید ابتدا مطالب فصل ۳ را تایپ نموده و سپس مطالب فصل ۱ را تایپ کنید. 
\subsubsection{مراجع}
برای وارد کردن مراجع \پ کافی است فایل 
\lr{MyReferences.bib}
را باز کرده و مراجع خود را به شکل اقلام نمونهٔ داخل آن، وارد کنید.  سپس از \lr{bibtex} برای تولید مراجع با قالب مناسب استفاده نمائید. برای توضیحات بیشتر بخش \ref{Sec:Ref} از پیوست \ref{App:latexIntro} و نیز پیوست \ref{App:RefMan} را ببینید.

\subsubsection{واژه‌نامه فارسی به انگلیسی و برعکس}
برای وارد کردن واژه‌نامه فارسی به انگلیسی و برعکس، چنانچه کاربر مبتدی هستید بهتر است مانند روش بکار رفته در فایل‌های 
\lr{dicfa2en.tex}
و
\lr{dicen2fa.tex}
عمل کنید. اما چنانچه کاربر پیشرفته هستید، بهتر است از بسته
\lr{glossaries}
استفاده کنید. راهنمای این بسته را می‌توانید به راحتی و با یک جستجوی ساده در اینترنت پیدا کنید.
\subsubsection{نمایه}
برای وارد کردن نمایه، باید از 
\lr{xindy}
استفاده کنید. 
%زیرا 
%\lr{MakeIndex}
%با حروف «گ»، «چ»، «پ»، «ژ» و «ک» مشکل دارد و ترتیب الفبایی این حروف را رعایت نمی‌کند. همچنین، فاصله بین هر گروه از کلمات در 
%\lr{MakeIndex}،
%به درستی رعایت نمی‌شود که باعث زشت شدن حروف‌چینی این قسمت می‌شود. 
راهنمای چگونگی کار با 
\lr{xindy} 
را می‌توانید در ویکی پارسی‌لاتک و یا مثالهای موجود در دی‌وی‌دی «مجموعه پارسی‌لاتک»، پیدا کنید.

\subsection{اگر سوالی داشتم، از کی بپرسم؟}
برای پرسیدن سوال‌های خود موقع حروف‌چینی با زی‌پرشین، می‌توانید به
\href{http://qa.parsilatex.com}{سایت پرسش و پاسخ پارسی‌لاتک}%
\LTRfootnote{http://qa.parsilatex.com}
یا
\href{http://forum.parsilatex.com}{بایگانی تالارگفتگوی قدیمی پارسی‌لاتک}%
\LTRfootnote{http://forum.parsilatex.com}
مراجعه کنید. شما هم می‌توانید روزی به سوال‌های دیگران در اینترنت جواب دهید.
بستهٔ زی‌پرشین و بسیاری از بسته‌های مرتبط با آن مانند
\lr{bidi} و
\lr{Persian-bib}،
مجموعه پارسی‌لاتک، مثالهای مختلف موجود در آن، قالب پایان‌نامه دانشگاههای مختلف و سایت پارسی‌لاتک همه به صورت داوطلبانه توسط افراد گروه پارسی‌لاتک و گروه
\lr{Persian TeX}
و بدون هیچ کمک مالی انجام شده‌اند. کار اصلی نوشتن و توسعه زی‌پرشین توسط آقای وفا خلیقی انجام شده است که این کار بزرگ را به انجام رساندند.
اگر مایل به کمک به گروه پارسی‌لاتک هستید به سایت این گروه مراجعه فرمایید:
\begin{center}
	\url{http://www.parsilatex.com}
\end{center}

\section{محتویات فصل اول یک پایان‌نامه}
فصل اول یک پایان‌نامه باید به مقدمه یا کلیات تحقیق بپردازد.
هدف از فصل مقدمه%
\LTRfootnote{Introduction}،
شرح مختصر مسأله تحقیق، اهمیت و انگیزه محقق از پرداختن به آن موضوع، بهمراه اشاره‌ای كوتاه به روش و مراحل تحقیق است. مقدمه، اولين فصل از ساختار اصلی \پ بوده و زمینه اطلاعاتی لازم را برای خواننده فراهم می‌آورد. در طول مقدمه باید سعی شود موضوع تحقیق با زبانی روشن، ساده و بطور عمیق و هدفمند به خواننده معرفی شود. این فصل باید خواننده را مجذوب و اهميت موضوع تحقيق را آشکار سازد. در مقدمه باید با ارائهٔ سوابق، شواهد تحقيقی و اطلاعات موجود (با ذکر منبع) با روشی منظم، منطقی و هدف‌دار، خواننده را جهت داد و به سوی راه حل مورد نظر هدايت کرد. مقدمه مناسب‌ترين جا برای ارائهٔ اختصارات و بعضی توضيحات کلی است، توضيحاتی که شايد نتوان در مباحث ديگر آنها را شرح داد.

مقدمه، یکی از ارکان اساسی و اصلی پایان نامه است که مهمترین قسمت‌های آن عبارتند از: 


\section{جمع‌بندی}!
و
\verb!% !TeX root=main.tex
\chapter{مروری بر مطالعات انجام شده}
%\thispagestyle{empty} 
\section{مقدمه}
هدف از اين فصل که با عنوان‌های  «مروری بر ادبیات موضوع%
\LTRfootnote{Literature Review}»،
«مروری بر منابع» و يا «مروری بر پیشینه تحقیق%
\LTRfootnote{Background Research}»
معرفی می‌شود، بررسی و طبقه‌بندی یافته‌های تحقیقات دیگر محققان در سطح دنیا و تعیین و شناسایی خلأهای تحقیقاتی است. آنچه را که تحقیق شما به دانش موجود اضافه می‌کند، مشخص کنید. طرح پیشینه تحقیق%
\LTRfootnote{Background Information}
یک مرور محققانه است و تا آنجا باید پیش برود که پیش‌زمینهٔ تاریخی مناسبی از تحقیق را بیان کند و جایگاه تحقیق فعلی را در میان آثار پیشین نشان دهد. برای این منظور منابع مرتبط با تحقیق را بررسی کنید، البته نه آنچنان گسترده که کل پیشینه تاریخی بحث را در برگیرد. برای نوشتن این بخش:
\begin{itemize}
	\item
	دانستنی‌های موجود و پیش‌زمینهٔ تاریخی و وضعیت کنونی موضوع را چنان بیان کنید که خواننده بدون مراجعه به منابع پیشین، نتایج حاصل از مطالعات قبلی را درک و ارزیابی کند.
	\item
	نشان دهید که بر موضوع احاطه دارید. پرسش تحقیق را همراه بحث و جدل‌ها و مسائل مطرح شده بیان کنید و مهم‌ترین تحقیق‌های انجام شده در این زمینه را معرفی نمائید.
	\item
	ابتدا مطالب عمومی‌تر و سپس پژوهش‌های مشابه با کار خود را معرفی کرده و نشان دهید که تحقیق شما از چه جنبه‌ای با کار دیگران تشابه یا تفاوت دارد.
	\item
	اگر کارهای قبلی را خلاصه کرده‌اید، از پرداختن به جزئیات غیرضروری بپرهیزید. در عوض، بر یافته‌ها و مسائل روش‌شناختی مرتبط و نتایج اصلی تأکید کنید و اگر بررسی‌ها و منابع مروری عمومی دربارهٔ موضوع موجود است، خواننده را به آنها ارجاع دهید.
\end{itemize}

\section{تعاریف، اصول و مبانی نظری}
این قسمت ارائهٔ خلاصه‌ای از دانش کلاسیک موضوع است. این بخش الزامی نیست و بستگی به نظر استاد راهنما دارد.

\section{مروری بر ادبیات موضوع}
در این قسمت باید به کارهای مشابه دیگران در گذشته اشاره کرد و وزن بیشتر این قسمت بهتر است به مقالات ژورنالی سال‌های اخیر (۲ تا ۳ سال) تخصیص داده شود. به نتایج کارهای دیگران با ذکر دقیق مراجع باید اشاره شده و جایگاه و تفاوت تحقیق شما نیز با کارهای دیگران مشخص شود. استفاده از مقالات ژورنال‌های معتبر در دو یا سه سال اخیر، می‌تواند به اعتبار کار شما بیافزاید.

\section{نتیجه‌گیری}
‌در نتیجه‌گیری آخر این فصل، با توجه به بررسی انجام شده بر روی مراجع تحقيق، بخش‌های قابل گسترش و تحقیق در آن حیطه و چشم‌اندازهای تحقیق مورد بررسی قرار می‌گیرند.	در برخی از تحقیقات، نتیجه نهایی فصل روش تحقیق، ارائهٔ یک چارچوب کار تحقیقی 
\lr{(research framework)}
است.!
را در فایل 
\lr{main.tex}،
غیرفعال%
\footnote{
برای غیرفعال کردن یک دستور، کافی است در ابتدای آن، علامت درصد انگلیسی (\%) بگذارید.
}
 کنید. در غیر این صورت، ابتدا مطالب دو فصل اول پردازش شده و سپس مطالب فصل ۳ پردازش می‌شود که این کار باعث طولانی شدن زمان پردازش می‌گردد. هر زمان که خروجی کل \پ را خواستید، تمام فصل‌ها را دوباره در
\lr{main.tex}
فعال نمائید.
بدیهتاً لازم نیست فصل‌های \پ را به ترتیب تایپ کنید. مثلاً می‌توانید ابتدا مطالب فصل ۳ را تایپ نموده و سپس مطالب فصل ۱ را تایپ کنید. 
\subsubsection{مراجع}
برای وارد کردن مراجع \پ کافی است فایل 
\lr{MyReferences.bib}
را باز کرده و مراجع خود را به شکل اقلام نمونهٔ داخل آن، وارد کنید.  سپس از \lr{bibtex} برای تولید مراجع با قالب مناسب استفاده نمائید. برای توضیحات بیشتر بخش \ref{Sec:Ref} از پیوست \ref{App:latexIntro} و نیز پیوست \ref{App:RefMan} را ببینید.

\subsubsection{واژه‌نامه فارسی به انگلیسی و برعکس}
برای وارد کردن واژه‌نامه فارسی به انگلیسی و برعکس، چنانچه کاربر مبتدی هستید بهتر است مانند روش بکار رفته در فایل‌های 
\lr{dicfa2en.tex}
و
\lr{dicen2fa.tex}
عمل کنید. اما چنانچه کاربر پیشرفته هستید، بهتر است از بسته
\lr{glossaries}
استفاده کنید. راهنمای این بسته را می‌توانید به راحتی و با یک جستجوی ساده در اینترنت پیدا کنید.
\subsubsection{نمایه}
برای وارد کردن نمایه، باید از 
\lr{xindy}
استفاده کنید. 
%زیرا 
%\lr{MakeIndex}
%با حروف «گ»، «چ»، «پ»، «ژ» و «ک» مشکل دارد و ترتیب الفبایی این حروف را رعایت نمی‌کند. همچنین، فاصله بین هر گروه از کلمات در 
%\lr{MakeIndex}،
%به درستی رعایت نمی‌شود که باعث زشت شدن حروف‌چینی این قسمت می‌شود. 
راهنمای چگونگی کار با 
\lr{xindy} 
را می‌توانید در ویکی پارسی‌لاتک و یا مثالهای موجود در دی‌وی‌دی «مجموعه پارسی‌لاتک»، پیدا کنید.

\subsection{اگر سوالی داشتم، از کی بپرسم؟}
برای پرسیدن سوال‌های خود موقع حروف‌چینی با زی‌پرشین، می‌توانید به
\href{http://qa.parsilatex.com}{سایت پرسش و پاسخ پارسی‌لاتک}%
\LTRfootnote{http://qa.parsilatex.com}
یا
\href{http://forum.parsilatex.com}{بایگانی تالارگفتگوی قدیمی پارسی‌لاتک}%
\LTRfootnote{http://forum.parsilatex.com}
مراجعه کنید. شما هم می‌توانید روزی به سوال‌های دیگران در اینترنت جواب دهید.
بستهٔ زی‌پرشین و بسیاری از بسته‌های مرتبط با آن مانند
\lr{bidi} و
\lr{Persian-bib}،
مجموعه پارسی‌لاتک، مثالهای مختلف موجود در آن، قالب پایان‌نامه دانشگاههای مختلف و سایت پارسی‌لاتک همه به صورت داوطلبانه توسط افراد گروه پارسی‌لاتک و گروه
\lr{Persian TeX}
و بدون هیچ کمک مالی انجام شده‌اند. کار اصلی نوشتن و توسعه زی‌پرشین توسط آقای وفا خلیقی انجام شده است که این کار بزرگ را به انجام رساندند.
اگر مایل به کمک به گروه پارسی‌لاتک هستید به سایت این گروه مراجعه فرمایید:
\begin{center}
	\url{http://www.parsilatex.com}
\end{center}

\section{محتویات فصل اول یک پایان‌نامه}
فصل اول یک پایان‌نامه باید به مقدمه یا کلیات تحقیق بپردازد.
هدف از فصل مقدمه%
\LTRfootnote{Introduction}،
شرح مختصر مسأله تحقیق، اهمیت و انگیزه محقق از پرداختن به آن موضوع، بهمراه اشاره‌ای كوتاه به روش و مراحل تحقیق است. مقدمه، اولين فصل از ساختار اصلی \پ بوده و زمینه اطلاعاتی لازم را برای خواننده فراهم می‌آورد. در طول مقدمه باید سعی شود موضوع تحقیق با زبانی روشن، ساده و بطور عمیق و هدفمند به خواننده معرفی شود. این فصل باید خواننده را مجذوب و اهميت موضوع تحقيق را آشکار سازد. در مقدمه باید با ارائهٔ سوابق، شواهد تحقيقی و اطلاعات موجود (با ذکر منبع) با روشی منظم، منطقی و هدف‌دار، خواننده را جهت داد و به سوی راه حل مورد نظر هدايت کرد. مقدمه مناسب‌ترين جا برای ارائهٔ اختصارات و بعضی توضيحات کلی است، توضيحاتی که شايد نتوان در مباحث ديگر آنها را شرح داد.

مقدمه، یکی از ارکان اساسی و اصلی پایان نامه است که مهمترین قسمت‌های آن عبارتند از: 


\section{جمع‌بندی}!
و
\verb!% !TeX root=main.tex
\chapter{مروری بر مطالعات انجام شده}
%\thispagestyle{empty} 
\section{مقدمه}
هدف از اين فصل که با عنوان‌های  «مروری بر ادبیات موضوع%
\LTRfootnote{Literature Review}»،
«مروری بر منابع» و يا «مروری بر پیشینه تحقیق%
\LTRfootnote{Background Research}»
معرفی می‌شود، بررسی و طبقه‌بندی یافته‌های تحقیقات دیگر محققان در سطح دنیا و تعیین و شناسایی خلأهای تحقیقاتی است. آنچه را که تحقیق شما به دانش موجود اضافه می‌کند، مشخص کنید. طرح پیشینه تحقیق%
\LTRfootnote{Background Information}
یک مرور محققانه است و تا آنجا باید پیش برود که پیش‌زمینهٔ تاریخی مناسبی از تحقیق را بیان کند و جایگاه تحقیق فعلی را در میان آثار پیشین نشان دهد. برای این منظور منابع مرتبط با تحقیق را بررسی کنید، البته نه آنچنان گسترده که کل پیشینه تاریخی بحث را در برگیرد. برای نوشتن این بخش:
\begin{itemize}
	\item
	دانستنی‌های موجود و پیش‌زمینهٔ تاریخی و وضعیت کنونی موضوع را چنان بیان کنید که خواننده بدون مراجعه به منابع پیشین، نتایج حاصل از مطالعات قبلی را درک و ارزیابی کند.
	\item
	نشان دهید که بر موضوع احاطه دارید. پرسش تحقیق را همراه بحث و جدل‌ها و مسائل مطرح شده بیان کنید و مهم‌ترین تحقیق‌های انجام شده در این زمینه را معرفی نمائید.
	\item
	ابتدا مطالب عمومی‌تر و سپس پژوهش‌های مشابه با کار خود را معرفی کرده و نشان دهید که تحقیق شما از چه جنبه‌ای با کار دیگران تشابه یا تفاوت دارد.
	\item
	اگر کارهای قبلی را خلاصه کرده‌اید، از پرداختن به جزئیات غیرضروری بپرهیزید. در عوض، بر یافته‌ها و مسائل روش‌شناختی مرتبط و نتایج اصلی تأکید کنید و اگر بررسی‌ها و منابع مروری عمومی دربارهٔ موضوع موجود است، خواننده را به آنها ارجاع دهید.
\end{itemize}

\section{تعاریف، اصول و مبانی نظری}
این قسمت ارائهٔ خلاصه‌ای از دانش کلاسیک موضوع است. این بخش الزامی نیست و بستگی به نظر استاد راهنما دارد.

\section{مروری بر ادبیات موضوع}
در این قسمت باید به کارهای مشابه دیگران در گذشته اشاره کرد و وزن بیشتر این قسمت بهتر است به مقالات ژورنالی سال‌های اخیر (۲ تا ۳ سال) تخصیص داده شود. به نتایج کارهای دیگران با ذکر دقیق مراجع باید اشاره شده و جایگاه و تفاوت تحقیق شما نیز با کارهای دیگران مشخص شود. استفاده از مقالات ژورنال‌های معتبر در دو یا سه سال اخیر، می‌تواند به اعتبار کار شما بیافزاید.

\section{نتیجه‌گیری}
‌در نتیجه‌گیری آخر این فصل، با توجه به بررسی انجام شده بر روی مراجع تحقيق، بخش‌های قابل گسترش و تحقیق در آن حیطه و چشم‌اندازهای تحقیق مورد بررسی قرار می‌گیرند.	در برخی از تحقیقات، نتیجه نهایی فصل روش تحقیق، ارائهٔ یک چارچوب کار تحقیقی 
\lr{(research framework)}
است.!
را در فایل 
\lr{main.tex}،
غیرفعال%
\footnote{
برای غیرفعال کردن یک دستور، کافی است در ابتدای آن، علامت درصد انگلیسی (\%) بگذارید.
}
 کنید. در غیر این صورت، ابتدا مطالب دو فصل اول  پردازش شده و سپس مطالب فصل ۳ پردازش می‌شود که این کار باعث طولانی شدن زمان پردازش می‌گردد. هر زمان که خروجی کل \پ را خواستید، تمام فصل‌ها را دوباره در
\lr{main.tex}
فعال نمائید.

\subsubsection{مراجع}
برای وارد کردن مراجع \پ خود، کافی است فایل 
\lr{MyReferences.bib}
را باز کرده و مراجع خود را مانند مراجع داخل آن، وارد کنید.  سپس از \lr{bibtex} برای تولید مراجع با قالب مناسب استفاده کنید. برای توضیحات بیشتر بخش \ref{Sec:Ref} و پیوست \ref{App:RefMan} را ببینید.


\subsubsection{واژه‌نامه فارسی به انگلیسی و برعکس}
برای وارد کردن واژه‌نامه فارسی به انگلیسی و برعکس، چنانچه کاربر مبتدی هستید، بهتر است مانند روش بکار رفته در فایل‌های 
\lr{dicfa2en}
و
\lr{dicen2fa}
عمل کنید. اما چنانچه کاربر پیشرفته هستید، بهتر است از بسته
\lr{glossaries}
استفاده کنید. راهنمای این بسته را می‌توانید به راحتی و با یک جستجوی ساده در اینترنت پیدا کنید.
\subsubsection{نمایه}
برای وارد کردن نمایه، باید از 
\lr{xindy}
استفاده کنید. 
%زیرا 
%\lr{MakeIndex}
%با حروف «گ»، «چ»، «پ»، «ژ» و «ک» مشکل دارد و ترتیب الفبایی این حروف را رعایت نمی‌کند. همچنین، فاصله بین هر گروه از کلمات در 
%\lr{MakeIndex}،
%به درستی رعایت نمی‌شود که باعث زشت شدن حروف‌چینی این قسمت می‌شود. 
راهنمای چگونگی کار با 
\lr{xindy} 
را می‌توانید در تالار گفتگوی پارسی‌لاتک و یا مثالهای موجود در مجموعه پارسی‌لاتک، پیدا کنید.

\subsection{اگر سوالی داشتم، از کی بپرسم؟}
برای پرسیدن سوال‌های خود موقع حروف‌چینی با زی‌پرشین،  می‌توانید به
 \href{http://forum.parsilatex.com}{تالار گفتگوی پارسی‌لاتک}%
\LTRfootnote{http://forum.parsilatex.com}
مراجعه کنید. شما هم می‌توانید روزی به سوال‌های دیگران در این تالار، جواب بدهید.
    
\subsection{جمع‌بندی}
بسته‌ی زی‌پرشین و بسیاری بسته‌های مرتبط با آن مانند \lr{bidi} و \lr{Persian-bib}، مجموعه پارسی‌لاتک، مثالهای مختلف موجود در آن، استیلهای مختلف پایان‌نامه دانشگاههای مختلف، سایت پارسی‌لاتک همه به صورت داوطلبانه توسط افراد گروه پارسی‌لاتک و بدون هیچ کمک مالی انجام شده‌اند. کار اصلی نوشتن و توسعه زی‌پرشین توسط آقای وفا خلیقی انجام شده است که این کار بزرگ را به انجام رساندند.
اگر مایل به کمک مالی به گروه پارسی‌لاتک هستید کمک‌های مالی خود را به  شماره حساب 
زیر نزد بانک ملی، به نام هادی صفی‌اقدم واریز نمایید:
\begin{center}
شماره حساب: ۰۱۰۱۲۰۰۰۷۰۰۰۳

شماره کارت: 
\lr{6037-9910-4168-7363}

شماره شبا: 
\lr{IR72-0170-0000-0010-1200-0700-03}
\end{center}
لطفاً پس از واریز وجه، موضوع را از طریق ایمیل به آقای صفی‌اقدم اطلاع دهید (\lr{hadi.safiaghdam@gmail.com}).
\section{محتویات فصل اول یک پایان‌نامه}