% !TeX root=main.tex
\chapter{بحث و نتیجه‌گیری}
%\thispagestyle{empty} 
\section{مقدمه}
تاکنون شما در پایان‌نامه‌ای که مشغول نوشتن آن هستید، پاسخ چهار سؤال را داده‌اید:
\begin{itemize}
	\item
	چرا تحقیق را انجام دادید؟ (مقدمه)
	\item
	دیگران در این زمینه‌ چه کارهایی کرده‌اند و تمایز کار شما با آنها؟ (مرور ادبیات)
	\item
	چگونه تحقیق را انجام دادید؟ (روش‌ها)
	\item
	چه از تحقیق به دست آوردید؟ (یافته‌ها)
\end{itemize}
حال زمان آن فرا رسیده که با توجه به تمامی مطالب ذکر شده، در نهایت به سؤال آخر پاسخ دهید:
\begin{itemize}
	\item
	چه برداشتی از یافته‌های تحقیق کردید؟ (نتیجه‌گیری)
\end{itemize}
در واقع در این بخش، هدف، پاسخ به این سوال است که چه برداشتی از یافته‌ها کردید و این یافته‌ها چه فایده‌ای دارند؟

نتیجه‌گیری مختصری بنویسید. ارائهٔ داده‌ها، نتایج و یافته‌ها در فصل چهارم ارائه می‌شود. در این فصل تفاوت، تضاد یا تطابق بین نتایج تحقیق با نتایج دیگر محققان باید ذکر شود.
\emph{تفسیر و تحلیل نتایج نباید بر اساس حدس و گمان باشد}،
بلکه باید
\textbf{برمبنای نتایج عملی استخراج‌شده}
از تحقیق و یا
\textbf{استناد به تحقیقات دیگران}
باشد.
با توجه به حجم و ماهیت تحقیق و با صلاحدید استاد راهنما، این فصل می‌تواند تحت عنوانی دیگر بیاید یا به دو فصل جداگانه با عناوین مناسب، تفکیک شود. اين فصل فقط باید به جمع‌بندی دست‌آوردهای فصل‌های سوم و چهارم محدود و از ذکر موارد جدید در آن خودداری شود. در عنوان این فصل، به جای کلمهٔ «تفسیر» می‌توان از واژگان «بحث» و «تحلیل» هم استفاده کرد. این فصل شاید مهم‌ترین فصل پایان‌نامه باشد.

در این فصل خلاصه‌ای از یافته‌های تحقیق جاری ارائه می‌شود. این فصل می‌تواند حاوی یک مقدمه، شامل مروری اجمالی بر مراحل انجام تحقیق باشد (حدود یک صفحه). مطالب پاراگراف‌بندی شود و هر پاراگراف به یک موضوع مستقل اختصاص یابد. فقط به ارائهٔ یافته‌ها و دست‌آوردها بسنده شود و
\emph{از تعمیم بی‌مورد نتایج خودداری شود.}
تا حد امکان از ارائهٔ 
\emph{جداول و نمودارها در این فصل اجتناب شود.}
از ارائهٔ 
\emph{عناوین کلی}
در حوزهٔ تحقیق و قسمت پيشنهاد تحقیقات آتی خودداری شود و کاملاً در چارچوب و زمينهٔ مربوط به تحقیق جاری باشد. این فصل حدود ۱۰-۱۵ صفحه است.

\section{محتوا}
به ترتیب شامل موارد زیر است:

\subsection{جمع‌بندی}
خلاصه‌ای از تمام یافته‌ها و دست‌آوردهای تحقیق جاری است.

\subsection{نوآوری}
این قسمت، نوآوری تحقیق را بر اساس یافته‌های آن تشریح می‌کند. که دارای دو بخش اصلی است:
\begin{enumerate}
	\item
	نوآوری تئوری، یعنی تمایز تئوریک کار با کارهای محققین قبلی.
	\item
	نوآوری عملی، یعنی توصیه‌های محقق به صنعت برای بهبود بخشیدن به کارها، بر اساس یافته‌های تحقیق.
\end{enumerate}

\subsection{پیشنهادها}
این بخش، عناوین و موضوعات پیشنهادی را برای تحقیقات آتی،
\emph{بیشتر در زمينهٔ مورد بحث در آينده}
ارائه می‌کند.

\subsection{محدودیت‌ها}
در اینجا انواع محدودیت‌های تحقیق تشریح می‌شوند؛ از جمله، محدودیت‌هایی که کنترل آن از عهده محقق خارج است، مانند انتخاب نوع یافته‌ها؛ و همچنین دیگر محدودیت‌هایی که کنترل آن در دست محقق است، مانند موضوع و محل تحقیق و ... . تأثیر این محدودیت‌ها بر یافته‌های تحقیق در این قسمت شرح داده می‌شوند.