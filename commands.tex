% در این فایل، دستورها و تنظیمات مورد نیاز، آورده شده است.
%-------------------------------------------------------------------------------------------------------------------

% در ورژن جدید زی‌پرشین برای تایپ متن‌های ریاضی، این سه بسته، حتماً باید فراخوانی شود
\usepackage{amsthm,amssymb,amsmath}
% بسته‌ای برای تنطیم حاشیه‌های بالا، پایین، چپ و راست صفحه
\usepackage[top=40mm, bottom=40mm, left=25mm, right=35mm]{geometry}
% بسته‌‌ای برای ظاهر شدن شکل‌ها و تصاویر متن
\usepackage{graphicx}
% بسته‌ای برای رسم کادر
\usepackage{framed} 
% بسته‌‌ای برای چاپ شدن خودکار تعداد صفحات در صفحه «معرفی پایان‌نامه»
\usepackage{lastpage}
% بسته‌ و دستوراتی برای ایجاد لینک‌های رنگی با امکان جهش
\usepackage[unicode=true,pagebackref=false,colorlinks,linkcolor=blue,citecolor=blue]{hyperref}
% چنانچه قصد پرینت گرفتن نوشته خود را دارید، خط بالا را غیرفعال و  از دستور زیر استفاده کنید چون در صورت استفاده از دستور زیر‌‌، 
% لینک‌ها به رنگ سیاه ظاهر خواهند شد که برای پرینت گرفتن، مناسب‌تر است
%\usepackage[pagebackref=false]{hyperref}
% بسته‌ لازم برای تنظیم سربرگ‌ها
\usepackage{fancyhdr}
%
%\usepackage{enumitem}
\usepackage{setspace}
\usepackage{algorithm}
\usepackage{algorithmic}

%بسته‌های لازم برای رسم بهتر جداول
\usepackage{tabulary}
\usepackage{tabularx}

\usepackage[export]{adjustbox}
\usepackage{subfigure}
\usepackage[subfigure]{tocloft}

% بسته‌ای برای رسم دیاگرام‌ها و نیز صفحه مالکیت اثر
\usepackage{tikz}

% بسته‌ای برای ظاهر شدن «مراجع» و «نمایه» در فهرست مطالب
\usepackage[nottoc]{tocbibind}
% دستورات مربوط به ایجاد نمایه
\usepackage{makeidx}
\makeindex
%%%%%%%%%%%%%%%%%%%%%%%%%%
% فراخوانی بسته زی‌پرشین و تعریف قلم فارسی و انگلیسی
\usepackage[extrafootnotefeatures]{xepersian}
\settextfont[BoldFont={IRLotusICEE_Bold.ttf}, BoldItalicFont={IRLotusICEE_BoldIranic.ttf}, ItalicFont={IRLotusICEE_Iranic.ttf},Scale=1.2]{IRLotusICEE.ttf}

\setlatintextfont[Scale=1]{Liberation Serif}% or FreeSerif
%%%%%%%%%%%%%%%%%%%%%%%%%%
% چنانچه می‌خواهید اعداد در فرمول‌ها، انگلیسی باشد، خط زیر را غیرفعال کنید
\setdigitfont[Scale=1.2]{IRLotusICEE.ttf}
%%%%%%%%%%%%%%%%%%%%%%%%%%
% تعریف قلم‌های فارسی و انگلیسی اضافی برای استفاده در بعضی از قسمت‌های متن
\defpersianfont\titlefont[Scale=1]{IRTitr.ttf}
\setiranicfont[Scale=1.3]{IRLotusICEE_Iranic.ttf}				% ایرانیک، خوابیده به چپ
% \defpersianfont\nastaliq[Scale=1.2]{IranNastaliq}
\setmathsfdigitfont{IRTitr.ttf}
%%%%%%%%%%%%%%%%%%%%%%%%%%
% دستوری برای حذف کلمه «چکیده»
\renewcommand{\abstractname}{}
% دستوری برای حذف کلمه «abstract»
%\renewcommand{\latinabstract}{}
% دستوری برای تغییر نام کلمه «اثبات» به «برهان»
\renewcommand\proofname{\textbf{برهان}}
% دستوری برای تغییر نام کلمه «کتاب‌نامه» به «مراجع»
\renewcommand{\bibname}{مراجع}
% دستوری برای تعریف واژه‌نامه انگلیسی به فارسی
\newcommand\persiangloss[2]{#1\dotfill\lr{#2}\\}
% دستوری برای تعریف واژه‌نامه فارسی به انگلیسی 
\newcommand\englishgloss[2]{#2\dotfill\lr{#1}\\}
% تعریف دستور جدید «\پ» برای خلاصه‌نویسی جهت نوشتن عبارت «پروژه/پایان‌نامه/رساله»
\newcommand{\پ}{پروژه/پایان‌نامه/رساله }

%\newcommand\BackSlash{\char`\\}

%%%%%%%%%%%%%%%%%%%%%%%%%%
\SepMark{-}

% تعریف و نحوه ظاهر شدن عنوان قضیه‌ها، تعریف‌ها، مثال‌ها و ...
\theoremstyle{definition}
\newtheorem{definition}{تعریف}[section]
\theoremstyle{theorem}
\newtheorem{theorem}[definition]{قضیه}
\newtheorem{lemma}[definition]{لم}
\newtheorem{proposition}[definition]{گزاره}
\newtheorem{corollary}[definition]{نتیجه}
\newtheorem{remark}[definition]{ملاحظه}
\theoremstyle{definition}
\newtheorem{example}[definition]{مثال}

%\renewcommand{\theequation}{\thechapter-\arabic{equation}}
%\def\bibname{مراجع}
\numberwithin{algorithm}{chapter}
\def\listalgorithmname{فهرست الگوریتم‌ها}
\def\listfigurename{فهرست تصاویر}
\def\listtablename{فهرست جداول}

%%%%%%%%%%%%%%%%%%%%%%%%%%%%
% دستورهایی برای سفارشی کردن سربرگ صفحات
% \newcommand{\SetHeader}{
% \csname@twosidetrue\endcsname
% \pagestyle{fancy}
% \fancyhf{} 
% \fancyhead[OL,EL]{\thepage}
% \fancyhead[OR]{\small\rightmark}
% \fancyhead[ER]{\small\leftmark}
% \renewcommand{\chaptermark}[1]{%
% \markboth{\thechapter-\ #1}{}}
% }
%%%%%%%%%%%%5
%\def\MATtextbaseline{1.5}
%\renewcommand{\baselinestretch}{\MATtextbaseline}
\doublespacing
%%%%%%%%%%%%%%%%%%%%%%%%%%%%%
% دستوراتی برای اضافه کردن کلمه «فصل» در فهرست مطالب

\newlength\mylenprt
\newlength\mylenchp
\newlength\mylenapp

\renewcommand\cftpartpresnum{\partname~}
\renewcommand\cftchappresnum{\chaptername~}
\renewcommand\cftchapaftersnum{:}

\settowidth\mylenprt{\cftpartfont\cftpartpresnum\cftpartaftersnum}
\settowidth\mylenchp{\cftchapfont\cftchappresnum\cftchapaftersnum}
\settowidth\mylenapp{\cftchapfont\appendixname~\cftchapaftersnum}
\addtolength\mylenprt{\cftpartnumwidth}
\addtolength\mylenchp{\cftchapnumwidth}
\addtolength\mylenapp{\cftchapnumwidth}

\setlength\cftpartnumwidth{\mylenprt}
\setlength\cftchapnumwidth{\mylenchp}	

\makeatletter
{\def\thebibliography#1{\chapter*{\refname\@mkboth
   {\uppercase{\refname}}{\uppercase{\refname}}}\list
   {[\arabic{enumi}]}{\settowidth\labelwidth{[#1]}
   \rightmargin\labelwidth
   \advance\rightmargin\labelsep
   \advance\rightmargin\bibindent
   \itemindent -\bibindent

   \listparindent \itemindent
   \parsep \z@
   \usecounter{enumi}}
   \def\newblock{}
   \sloppy
   \sfcode`\.=1000\relax}}
   
%اگر مایلید در شماره گذاری حرفی و ابجد به جای آ از الف استفاده شود دستورات زیر را فعال کنید.   
%\def\@Abjad#1{%
%  \ifcase#1\or الف\or ب\or ج\or د%
%           \or هـ\or و\or ز\or ح\or ط%
%           \or ی\or ک\or ل\or م\or ن%
%           \or س\or ع\or ف\or ص%
%           \or ق\or ر\or ش\or ت\or ث%
%            \or خ\or ذ\or ض\or ظ\or غ%
%            \else\@ctrerr\fi}
%
% \def\abj@num@i#1{%
%   \ifcase#1\or الف\or ب\or ج\or د%
%            \or هـ‍\or و\or ز\or ح\or ط\fi

%   \ifnum#1=\z@\abjad@zero\fi}   
%  
%   \def\@harfi#1{\ifcase#1\or الف\or ب\or پ\or ت\or ث\or

% ج\or چ\or ح\or خ\or د\or ذ\or ر\or ز\or ژ\or س\or ش\or ص\or ض\or ط\or ظ\or ع\or غ\or

% ف\or ق\or ک\or گ\or ل\or م\or ن\or و\or ه\or ی\else\@ctrerr\fi}

%
\makeatother

% for numbering subsubsections
\setcounter{secnumdepth}{3}
%to include subsubsections in the table of contents
\setcounter{tocdepth}{3}