% !TeX root=main.tex
% در این فایل، عنوان پایان‌نامه، مشخصات خود، متن تقدیمی‌، ستایش، سپاس‌گزاری و چکیده پایان‌نامه را به فارسی، وارد کنید.
% توجه داشته باشید که جدول حاوی مشخصات پروژه/پایان‌نامه/رساله و همچنین، مشخصات داخل آن، به طور خودکار، درج می‌شود.
%%%%%%%%%%%%%%%%%%%%%%%%%%%%%%%%%%%%
% دانشگاه خود را وارد کنید
\university{دانشگاه تهران}
% پردیس دانشگاهی خود را اگر نیاز است وارد کنید (مثال: فنی، علوم پایه، علوم انسانی و ...)
\college{پردیس دانشکده‌های فنی}
% دانشکده، آموزشکده و یا پژوهشکده  خود را وارد کنید
\faculty{دانشکده علوم مهندسی}
% گروه آموزشی خود را وارد کنید (در صورت نیاز)
\department{گروه الگوریتم‌ها و محاسبات}
% رشته تحصیلی خود را وارد کنید
\subject{مهندسی کامپیوتر}
% گرایش خود را وارد کنید
\field{الگوریتم‌ها و محاسبات}
% عنوان پایان‌نامه را وارد کنید
\title{نوشتن پروژه، پایان‌نامه و رساله با استفاده از کلاس 
tehran-thesis}
% نام استاد(ان) راهنما را وارد کنید
\firstsupervisor{استاد راهنمای اول}
\secondsupervisor{استاد راهنمای دوم}
% نام استاد(دان) مشاور را وارد کنید. چنانچه استاد مشاور ندارید، دستور پایین را غیرفعال کنید.
\firstadvisor{استاد مشاور اول}
%\secondadvisor{استاد مشاور دوم}
% نام دانشجو را وارد کنید
\name{سینا}
% نام خانوادگی دانشجو را وارد کنید
\surname{ممکن}
% شماره دانشجویی دانشجو را وارد کنید
\studentID{810893024}
% تاریخ پایان‌نامه را وارد کنید
\thesisdate{مرداد ۱۳۹۶}
% به صورت پیش‌فرض برای پایان‌نامه‌های کارشناسی تا دکترا به ترتیب از عبارات «پروژه»، «پایان‌نامه» و «رساله» استفاده می‌شود؛ اگر  نمی‌پسندید هر عنوانی را که مایلید در دستور زیر قرار داده و آنرا از حالت توضیح خارج کنید.
%\projectLabel{پایان‌نامه}

% به صورت پیش‌فرض برای عناوین مقاطع تحصیلی کارشناسی تا دکترا به ترتیب از عبارت «کارشناسی»، «کارشناسی ارشد» و «دکتری» استفاده می‌شود؛ اگر نمی‌پسندید هر عنوانی را که مایلید در دستور زیر قرار داده و آنرا از حالت توضیح خارج کنید.
%\degree{}

\titlePage
\besmPage
\titlePage
\davaranPage

%\vspace{.5cm}
%%%%%%%%%%%%%%%%%%%%%%%%%%%%%%%%%%%%%%%%%%%%%%%%%%%%%%
% در این قسمت اسامی اساتید راهنما، مشاور و داور باید به صورت دستی وارد شوند
%\renewcommand{\arraystretch}{1.2}
\begin{center}
\begin{tabular}{| p{8mm} | p{18mm} | p{.17\textwidth} |p{14mm}|p{.2\textwidth}|c|}
\hline
ردیف & مشخصات هیئت داوران & نام و نام خانوادگی & مرتبه \newline دانشگاهی &	دانشگاه یا مؤسسه & امضـــــــــــــا \\
\hline
۱ &	استاد راهنما & دکتر دارا معظمی & استاد & دانشگاه تهران & \\

۲ &	استاد راهنما & دکتر کاوه کاووسی & استادیار & دانشگاه تهران & \\
\hline
۳ & استاد مشاور & دکتر علی‌محمد بنائی‌مقدم & استادیار & دانشگاه تهران & \\
\hline
۴ & استاد داور داخلی & دکتر داور داخلی & دانشیار & دانشگاه تهران & \\
\hline
۵ & استاد مدعو & دکتر داور خارجی & دانشیار & دانشگاه داور خارجی & \\
\hline
۶ & نماینده تحصیلات تکمیلی دانشکده & دکتر نماینده & دانشیار & دانشگاه تهران & \\
\hline
\end{tabular}
\end{center}

\esalatPage
\mojavezPage

%%%%%%%%%%%%%%%%%%%%%%%%%%%%%%%%%%%%%%%%%%%%%%%%%%%%%%
% چنانچه مایل به چاپ صفحات «تقدیم»، «نیایش» و «سپاس‌گزاری» در خروجی نیستید، خط‌های زیر را با گذاشتن ٪  در ابتدای آنها غیرفعال کنید.

%% پایان‌نامه خود را تقدیم کنید! %%
\newpage
\thispagestyle{empty}
{\Large تقدیم به:}\\
\begin{flushleft}
{\huge
همسر و فرزندانم\\
\vspace{7mm}
و\\
\vspace{7mm}
پدر و مادرم
}
\end{flushleft}


%% قدردانی %%
%% ترجیحا با توجه به ذوق و سلیقه خود متن قدردانی را تغییر دهید.
\begin{acknowledgementpage}
سپاس خداوندگار حکیم را که با لطف بی‌کران خود، آدمی را به زیور عقل آراست.

در آغاز وظیفه‌  خود  می‌دانم از زحمات بی‌دریغ اساتید  راهنمای خود،  جناب آقای دکتر ... و ...، صمیمانه تشکر و  قدردانی کنم که در طول انجام این پایان‌نامه با نهایت صبوری همواره راهنما و مشوق من بودند و قطعاً بدون راهنمایی‌های ارزنده‌ ایشان، این مجموعه به انجام نمی‌رسید.

از جناب آقای دکتر ... که  زحمت مشاوره‌، بازبینی و تصحیح این پایان‌نامه را تقبل فرمودند کمال امتنان را دارم.

%از همکاری و مساعدت‌های دکتر ... مسئول تحصیلات تکمیلی و سایر کارکنان دانشکده بویژه سرکار خانم ... کمال تشکر را دارم.

با سپاس بی‌دریغ خدمت دوستان گران‌مایه‌ام، خانم‌ها ... و آقایان ... در آزمایشگاه ...، که با همفکری مرا صمیمانه و مشفقانه یاری داده‌اند.

و در پایان، بوسه می‌زنم بر دستان خداوندگاران مهر و مهربانی، پدر و مادر عزیزم و بعد از خدا، ستایش می‌کنم وجود مقدس‌شان را و تشکر می‌کنم از خانواده عزیزم به پاس عاطفه سرشار و گرمای امیدبخش وجودشان، که بهترین پشتیبان من بودند. 
% با استفاده از دستور زیر، امضای شما، به طور خودکار، درج می‌شود.
\signature 
\end{acknowledgementpage}
%%%%%%%%%%%%%%%%%%%%%%%%%%%%%%%%%%%%
% کلمات کلیدی پایان‌نامه را وارد کنید
\keywords{حداکثر ۵ کلمه یا عبارت، متناسب با عنوان، قالب پایان‌نامه، لاتک}
%چکیده پایان‌نامه را وارد کنید، برای ایجاد پاراگراف جدید از \\ استفاده کنید. اگر خط خالی دشته باشید، خطا خواهید گرفت.
\fa-abstract{
این راهنما، نمونه‌ای از قالبِ پروژه، پایان‌نامه و رسالهٔ دانشگاه تهران می‌باشد که با استفاده از کلاس 
\lr{tehran-thesis}
و بستهٔ زی‌پرشین در \lr{\LaTeX}{} تهیه شده است. این قالب به گونه‌ای طراحی شده است که مطابق با دستورالعمل نگارش و تدوین پایان‌نامه کارشناسی ارشد و دکتری، مورخ ۹۳/۰۶/۰۳ پردیس دانشکده‌های فنی دانشگاه تهران باشد و حروف‌چینی بسیاری از قسمت‌های آن، مطابق با استاندارد قالب‌های فارسی پایان‌نامه در لاتک، به طور خودکار انجام می‌شود.
\\
چكیده بخشی از پایان‌نامه است كه خواننده را به مطالعه آن علاقمند می‌كند و یا از آن می‌گریزاند. چكیده باید ترجیحاً‌ در یک صفحه باشد. در نگارش چکیده نکات زیر باید رعایت شود. متن چكيده باید مزین به کلمه‌ها و عبارات سليس، آشنا، بامعنی و روشن باشد. بگونه‌ای که با حدود ۳۰۰ تا ۵۰۰ کلمه بتواند خواننده را به خواندن پایان‌نامه راغب نماید. چكيده، جدای از پا‌یان‌نامه بايد به تنهایی گويا و مستقل باشد. در چکیده باید از ذکر منابع، اشاره به جداول و نمودارها اجتناب شود.تمیز بودن مطلب، نداشتن غلط‌های املایی یا دستور زبانی و رعایت دقت و تسلسل روند نگارش چکيده از نکات مهم دیگری است كه باید درنظر گرفته شود. در چکیده پایان‌نامه باید از درج مشخصات مربوط به پایان‌نامه خودداری شود.
چکیده باید منعکس‌کننده اصل موضوع باشد. در چکیده باید اهداف تحقیق مورد توجه قرار گیرد. تأکید روی اطلاعات تازه (یافته‌ها) و اصطلاحات جدید یا نظریه‌ها، فرضیه‌ها، نتایج و پیشنهادها متمركز شود. اگر در پایان‌نامه روش نوینی برای اولین بار ارائه می‌شود و تا به حال معمول نبوده است، با جزئیات بیشتری ذكر شود. شایان ذکر است چکیده فارسی و انگلیسی باید حتماً به تأیید استاد راهنما رسیده باشد.
\\
کلمات کلیدی در انتهای چکیده فارسی و انگلیسی آورده می‌شود. محتوای چکیده‌ها بر اساس موضوع و گرایش تحقیق طبقه‌بندی می‌شود و به همین جهت وجود کلمات شاخص و كلیدی، مراکز اطلاعاتی  را در طبقه‌بندی دقیق و سریع پایان‌نامه یاری می‌دهد. کلمات کلیدی، راهنمای نکات مهم موجود در پایان‌نامه هستند. بنابراین باید در حد امکان کلمه‌ها یا عباراتی انتخاب شود که ماهیت، محتوا و گرايش کار را به وضوح روشن نماید.
}

\abstractPage

\newpage\clearpage